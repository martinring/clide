

\documentclass[11pt,a4paper]{article}
\usepackage{isabelle,isabellesym}


% this should be the last package used
\usepackage{pdfsetup}

\begin{document}

\title{Functional Automata}
\author{Tobias Nipkow}
\maketitle

\begin{abstract}
This theory defines deterministic and nondeterministic automata in a
functional representation: the transition function/relation and the finality
predicate are just functions. Hence the state space may be infinite.  It is
shown how to convert regular expressions into such automata. A scanner
(generator) is implemented with the help of functional automata: the scanner
chops the input up into longest recognized substrings. Finally we also show
how to convert a certain subclass of functional automata (essentially the
finite deterministic ones) into regular sets.
\end{abstract}

\section{Overview}

The theories are structured as follows:
\begin{itemize}
\item Automata:
 \texttt{AutoProj}, \texttt{NA}, \texttt{NAe}, \texttt{DA}, \texttt{Automata}
\item Conversion of regular expressions into automata:\\
  \texttt{RegExp2NA}, \texttt{RegExp2NAe}, \texttt{AutoRegExp}.
\item Scanning: \texttt{MaxPrefix}, \texttt{MaxChop}, \texttt{AutoMaxChop}.
\end{itemize}
For a full description see \cite{Nipkow-TPHOLs98}.

In contrast to that paper, the latest version of the theories provides a
fully executable scanner generator. The non-executable bits (transitive
closure) have been eliminated by going from regular expressions directly to
nondeterministic automata, thus bypassing epsilon-moves.

Not described in the paper is the conversion of certain functional automata
(essentially the finite deterministic ones) into regular sets contained in
\texttt{RegSet\_of\_nat\_DA}.

% include generated text of all theories
\input{session}

\bibliographystyle{abbrv}
\bibliography{root}

\end{document}
