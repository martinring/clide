

\documentclass[11pt,a4paper]{article}
\usepackage{isabelle,isabellesym}

% further packages required for unusual symbols (see also isabellesym.sty)
% use only when needed
%\usepackage{amssymb}                  % for \<leadsto>, \<box>, \<diamond>,
                                       % \<sqsupset>, \<mho>, \<Join>, 
                                       % \<lhd>, \<lesssim>, \<greatersim>,
                                       % \<lessapprox>, \<greaterapprox>,
                                       % \<triangleq>, \<yen>, \<lozenge>
%\usepackage[greek,english]{babel}     % greek for \<euro>,
                                       % english for \<guillemotleft>, 
                                       %             \<guillemotright>
                                       % default language = last
%\usepackage[latin1]{inputenc}         % for \<onesuperior>, \<onequarter>,
                                       % \<twosuperior>, \<onehalf>,
                                       % \<threesuperior>, \<threequarters>,
                                       % \<degree>
%\usepackage[only,bigsqcap]{stmaryrd}  % for \<Sqinter>
%\usepackage{eufrak}                   % for \<AA> ... \<ZZ>, \<aa> ... \<zz>
                                       % (only needed if amssymb not used)
%\usepackage{textcomp}                 % for \<cent>, \<currency>

% this should be the last package used
\usepackage{pdfsetup}

% urls in roman style, theory text in math-similar italics
\urlstyle{rm}
\isabellestyle{it}


\begin{document}

\title{Abstract Hoare Logics}
\author{Tobias Nipkow}
\maketitle

\begin{abstract}
  These therories describe Hoare logics for a number of imperative
  language constructs, from while-loops to mutually
  recursive procedures. Both partial and total correctness are
  treated. In particular a proof system for total correctness of
  recursive procedures in the presence of unbounded nondeterminism is
  presented.
\end{abstract}

\tableofcontents

\section{Introduction}

These are the theories underlying the publications
\cite{Nipkow-MOD2001,Nipkow-CSL02}. They should be consulted for explanatory
text. The local variable declaration construct in \cite{Nipkow-MOD2001} has
been generalized; see Section~\ref{sec:lang}.

\input{session}

\bibliographystyle{abbrv}
\bibliography{root}

\end{document}
