

\documentclass[11pt,a4paper]{article}
\usepackage{isabelle,isabellesym}

% further packages required for unusual symbols (see also isabellesym.sty)
% use only when needed
%\usepackage{amssymb}                  % for \<leadsto>, \<box>, \<diamond>,
                                       % \<sqsupset>, \<mho>, \<Join>, 
                                       % \<lhd>, \<lesssim>, \<greatersim>,
                                       % \<lessapprox>, \<greaterapprox>,
                                       % \<triangleq>, \<yen>, \<lozenge>
%\usepackage[greek,english]{babel}     % greek for \<euro>,
                                       % english for \<guillemotleft>, 
                                       %             \<guillemotright>
                                       % default language = last
%\usepackage[latin1]{inputenc}         % for \<onesuperior>, \<onequarter>,
                                       % \<twosuperior>, \<onehalf>,
                                       % \<threesuperior>, \<threequarters>,
                                       % \<degree>
%\usepackage[only,bigsqcap]{stmaryrd}  % for \<Sqinter>
%\usepackage{eufrak}                   % for \<AA> ... \<ZZ>, \<aa> ... \<zz>
                                       % (only needed if amssymb not used)
%\usepackage{textcomp}                 % for \<cent>, \<currency>

% this should be the last package used
\usepackage{pdfsetup}

% urls in roman style, theory text in math-similar italics
\urlstyle{rm}
\isabellestyle{it}


\begin{document}

\title{Regular Sets, Expressions, Derivatives and Relation Algebra}
\author{Alexander Krauss, Tobias Nipkow,\\
  Chunhan Wu, Xingyuan Zhang and Christian Urban}
\maketitle

\begin{abstract}
This is a library of constructions on regular expressions and languages.  It
provides the operations of concatenation, Kleene star and left-quotients of
languages. A theory of derivatives and partial derivatives is
provided. Arden's lemma and finiteness of partial derivatives is
established. A simple regular expression matcher based on Brozowski's
derivatives is proved to be correct.  An executable equivalence checker for
regular expressions is verified; it does not need automata but works directly
on regular expressions. By mapping regular expressions to binary relations, an
automatic and complete proof method for (in)equalities of binary relations
over union, concatenation and (reflexive) transitive closure is obtained.

For an exposition of the equivalence checker for regular and relation
algebraic expressions see the paper by Krauss and Nipkow~\cite{KraussN-JAR}.

Extended regular expressions with complement and intersection
are also defined and an equivalence checker is provided.
\end{abstract}

\tableofcontents

% include generated text of all theories
\input{session}

\bibliographystyle{abbrv}
\bibliography{root}

\end{document}
