\documentclass[11pt,a4paper]{article}
\usepackage{isabelle,isabellesym}

% further packages required for unusual symbols (see also
% isabellesym.sty), use only when needed

\usepackage{amssymb}
  %for \<leadsto>, \<box>, \<diamond>, \<sqsupset>, \<mho>, \<Join>,
  %\<lhd>, \<lesssim>, \<greatersim>, \<lessapprox>, \<greaterapprox>,
  %\<triangleq>, \<yen>, \<lozenge>

\newcommand{\isasymcongruent}{\isamath{\equiv}}
\renewcommand{\isasymequiv}{\isamath{\triangleq}}

%\usepackage[greek,english]{babel}
  %option greek for \<euro>
  %option english (default language) for \<guillemotleft>, \<guillemotright>

\usepackage[latin1]{inputenc}
  %for \<onesuperior>, \<onequarter>, \<twosuperior>, \<onehalf>,
  %\<threesuperior>, \<threequarters>, \<degree>

%\usepackage[only,bigsqcap]{stmaryrd}
  %for \<Sqinter>

%\usepackage{eufrak}
  %for \<AA> ... \<ZZ>, \<aa> ... \<zz> (also included in amssymb)

%\usepackage{textcomp}
  %for \<cent>, \<currency>

% this should be the last package used
\usepackage{pdfsetup}

% urls in roman style, theory text in math-similar italics
\urlstyle{rm}
\isabellestyle{it}


\begin{document}

\title{The independence of Tarski's Euclidean axiom}
\author{T.~J.~M.~Makarios}
\maketitle

\begin{abstract}
Tarski's axioms of plane geometry are formalized and, using the standard real Cartesian model, shown to be consistent.
A substantial theory of the projective plane is developed.
Building on this theory, the Klein--Beltrami model of the hyperbolic plane is defined and shown to satisfy all of Tarski's axioms except his Euclidean axiom;
thus Tarski's Euclidean axiom is shown to be independent of his other axioms of plane geometry.

An earlier version of this work was the subject of the author's MSc thesis \cite{makarios}, which contains natural-language explanations of some of the more interesting proofs.
\end{abstract}

\tableofcontents

% generated text of all theories
\input{session}

% optional bibliography
\bibliographystyle{abbrv}
\bibliography{root}

\end{document}

%%% Local Variables:
%%% mode: latex
%%% TeX-master: t
%%% End:
