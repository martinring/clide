\documentclass[11pt,a4paper]{article}
\usepackage{isabelle,isabellesym}

% further packages required for unusual symbols (see also
% isabellesym.sty), use only when needed

%\usepackage{amssymb}
  %for \<leadsto>, \<box>, \<diamond>, \<sqsupset>, \<mho>, \<Join>,
  %\<lhd>, \<lesssim>, \<greatersim>, \<lessapprox>, \<greaterapprox>,
  %\<triangleq>, \<yen>, \<lozenge>

%\usepackage[greek,english]{babel}
  %option greek for \<euro>
  %option english (default language) for \<guillemotleft>, \<guillemotright>

\usepackage[latin1]{inputenc}
  %for \<onesuperior>, \<onequarter>, \<twosuperior>, \<onehalf>,
  %\<threesuperior>, \<threequarters>, \<degree>

\usepackage[only,bigsqcap]{stmaryrd}
  %for \<Sqinter>

%\usepackage{eufrak}
  %for \<AA> ... \<ZZ>, \<aa> ... \<zz> (also included in amssymb)

%\usepackage{textcomp}
  %for \<cent>, \<currency>

% this should be the last package used
\usepackage{pdfsetup}

% urls in roman style, theory text in math-similar italics
\urlstyle{rm}
\isabellestyle{it}


\begin{document}

\title{Pseudo-hoops}

\author{George Georgescu and Lauren\c tiu Leu\c stean and Viorel Preoteasa}

\maketitle

\begin{abstract}
Pseudo-hoops are algebraic structures introduced in \cite{bosbach:1969,bosbach:1970} 
by B. Bosbach under the name of complementary semigroups.
This is a formalization of the paper \cite{georgescu:leustean:preoteasa:2005}.
Following \cite{georgescu:leustean:preoteasa:2005}  we prove some properties 
of pseudo-hoops and we define the basic concepts of filter and normal filter. 
The lattice of normal filters is isomorphic with the lattice of congruences of 
a pseudo-hoop. We also study some important classes of pseudo-hoops. Bounded 
Wajsberg pseudo-hoops are equivalent to pseudo-Wajsberg algebras and bounded 
basic pseudo-hoops are equiv- alent to pseudo-BL algebras. Some examples of 
pseudo-hoops are given in the last section of the formalization.

\end{abstract}

\tableofcontents

\section{Overview}
Section 2 introduces some operations and their infix syntax.
Section 3 and 4 introduces some facts about residuated and complemented monoids.
Section 5 introduces the pseudo-hoops and some of their properties. Section 6
introduces filters and normal filters and proves that the lattice of normal 
filters and the lattice of congruences are isomorphic. Following 
\cite{ceterchi:2001}, section 7 introduces pseudo-Waisberg algebras and 
some of their properties. In Section 8 we investigate some classes of 
pseudo-hoops. Finally section 9 presents some examples of pseudo-hoops and
normal filters.

\parindent 0pt\parskip 0.5ex

% generated text of all theories
\input{session}

% optional bibliography
\bibliographystyle{abbrv}
\bibliography{root}

\end{document}

%%% Local Variables:
%%% mode: latex
%%% TeX-master: t
%%% End:
