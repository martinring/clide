\documentclass[11pt,a4paper]{article}
\usepackage{isabelle,isabellesym}
\usepackage{verbatim}

% further packages required for unusual symbols (see also isabellesym.sty)
% use only when needed
%\usepackage{amssymb}                  % for \<leadsto>, \<box>, \<diamond>,
                                       % \<sqsupset>, \<mho>, \<Join>, 
                                       % \<lhd>, \<lesssim>, \<greatersim>,
                                       % \<lessapprox>, \<greaterapprox>,
                                       % \<triangleq>, \<yen>, \<lozenge>
%\usepackage[greek,english]{babel}     % greek for \<euro>,
                                       % english for \<guillemotleft>, 
                                       %             \<guillemotright>
                                       % default language = last
%\usepackage[latin1]{inputenc}         % for \<onesuperior>, \<onequarter>,
                                       % \<twosuperior>, \<onehalf>,
                                       % \<threesuperior>, \<threequarters>,
                                       % \<degree>
%\usepackage[only,bigsqcap]{stmaryrd}  % for \<Sqinter>
%\usepackage{eufrak}                   % for \<AA> ... \<ZZ>, \<aa> ... \<zz>
                                       % (only needed if amssymb not used)
%\usepackage{textcomp}                 % for \<cent>, \<currency>

% this should be the last package used
\usepackage{pdfsetup}

% urls in roman style, theory text in math-similar italics
\urlstyle{rm}
\isabellestyle{it}


\begin{document}

\title{Instances of Schneider's generalized protocol\\
       of clock synchronization.}

\author{Damian Barsotti}
\maketitle

\begin{abstract}\noindent  
  Schneider \cite{schneider87understanding} generalizes a number of
  protocols for Byzantine fault-tolerant clock synchronization and
  presents a uniform proof for their correctness. In Schneider's
  schema, each processor maintains a local clock by periodically
  adjusting each value to one computed by a convergence function
  applied to the readings of all the clocks. Then, correctness of an
  algorithm, i.e. that the readings of two clocks at any time are
  within a fixed bound of each other, is based upon some conditions on
  the convergence function. To prove that a particular clock
  synchronization algorithm is correct it suffices to show that the
  convergence function used by the algorithm meets Schneider's
  conditions.

  Using the theorem prover Isabelle, we formalize the proofs that the
  convergence functions of two algorithms, namely, the Interactive
  Convergence Algorithm (ICA) of Lamport and Melliar-Smith
  \cite{lamport_cs} and the Fault-tolerant Midpoint algorithm of
  Lundelius-Lynch \cite{lynch_cs}, meet Schneider's conditions.
  Furthermore, we experiment on handling some parts of the proofs with
  fully automatic tools like ICS\cite{ics} and \mbox{CVC-lite}\cite{cvclite}.

  These theories are part of a joint work with Alwen Tiu and Leonor
  P. Nieto \cite{bars_leon_tiu}. In this work the correctness of
  Schneider schema was also verified using Isabelle (available at
  \url{http://afp.sourceforge.net/entries/GenClock.shtml}).

\end{abstract}

\tableofcontents

\parindent 0pt\parskip 0.5ex

% include generated text of all theories
\input{session}

\appendix

\section{CVC-lite and ICS proofs}

\subsection{Lemma abs\_distrib\_div}
\label{sec:abs_distrib_mult}

In the proof of the Fault-Tolerant Mid Point Algorithm we need to
prove this simple lemma:

\begin{isabellebody}%
\isamarkuptrue%
\isacommand{lemma}\ abs{\isacharunderscore}distrib{\isacharunderscore}div{\isacharcolon}\isanewline
\ \ {\isachardoublequote}{\isadigit{0}}\ {\isacharless}\ {\isacharparenleft}c{\isacharcolon}{\isacharcolon}real{\isacharparenright}\ \ {\isasymLongrightarrow}\ {\isasymbar}a\ {\isacharslash}\ c\ {\isacharminus}\ b\ {\isacharslash}\ c{\isasymbar}\ {\isacharequal}\ {\isasymbar}a\ {\isacharminus}\ b{\isasymbar}\ {\isacharslash}\ c{\isachardoublequote}\isanewline
\isamarkupfalse%
\end{isabellebody}%
It is not possible to prove this lemma in Isabelle using \emph{arith} nor
\emph{auto} tactics. Even if we added lemmas to the default simpset of
HOL. 

In the translation from Isabelle to ICS we need to change the division
by a multiplication because this tools do not accept formulas with this
arithmetic operator.  Moreover, to translate the absolute value we
define e constant for each application of that function.
In ICS it is proved automatically. 

File \verb|abs_distrib_mult.ics|:
\verbatiminput{abs_distrib_mult.ics}

It was not possible to find the proof in CVC-lite because the
formula is not linear. Two proofs where attempted. In the first one we
use lambda abstraction to define the absolute value. The second one is
the same translation that we do in ICS.

File \verb|abs_distrib_mult.cvc|:
\verbatiminput{abs_distrib_mult.cvc}

File \verb|abs_distrib_mult2.cvc|:
\verbatiminput{abs_distrib_mult2.cvc}

\subsection{Bound for Precision Enhancement property}
\label{sec:bound_prec_enh}

In order to prove Precision Enhancement for Lynch's algorithm we need
to prove that:

\begin{isabellebody}%
\ \ \ \ \ \isacommand{have}\ {\isachardoublequote}{\isasymbar}Max\
{\isacharparenleft}reduce\ f\ PR{\isacharparenright}\ {\isacharplus}\
Min\ {\isacharparenleft}reduce\ f\
PR{\isacharparenright}\ \ {\isacharplus}\ 
\isanewline
\ \ \ \ \ \ \ \ \ \ \ \ {\isacharminus}\
Max\
{\isacharparenleft}reduce\ g\ PR{\isacharparenright}\ {\isacharplus}\ {\isacharminus}\
Min\ {\isacharparenleft}reduce\ g\
PR{\isacharparenright}{\isasymbar}\ {\isacharless}{\isacharequal}\ y\
{\isacharplus}\ {\isadigit{2}}\ {\isacharasterisk}\
x{\isachardoublequote}
\end{isabellebody}%
This is the result of the whole theorem where we multiply by two both
sides of the inequality.

In order to do the proof we need to translate also the lemmas
\emph{uboundmax}, \emph{lboundmin}, \emph{same\_bound} (lemmas about
the existence of some bounds), the axiom \emph{constants\_ax} and the
assumptions of the theorem.

We make five different translations. In each one we where increasing
the amount of eliminated quantifiers. 



File \verb|bound_prec_enh4.cvc|:
\verbatiminput{bound_prec_enh4.cvc}

Note that we leave quantifiers in some assumptions.

In the next file we also try to do the proof with all quantifiers,
but CVC cannot find it.

File \verb|bound_prec_enh.cvc|:
\verbatiminput{bound_prec_enh.cvc}

We also try to do the proof removing all quantifiers and the proof
was successful.

File \verb|bound_prec_enh7.cvc|:
\verbatiminput{bound_prec_enh7.cvc}

From this last file we make the translation also for ICS adding a
constant for each application of the absolute value. In this case ICS
do not find the proof.

File \verb|bound_prec_enh.ics|:
\verbatiminput{bound_prec_enh.ics}

\subsection{Accuracy Preservation property}
\label{sec:accur_pres}

The proof of this property was successful in both tools. Even in
CVC-lite the proof was find without the need of removing the
quantifiers.

File \verb|accur_pres.cvc|:
\verbatiminput{accur_pres.cvc}

File \verb|accur_pres.ics|:
\verbatiminput{accur_pres.ics}


\bibliographystyle{abbrv}
\bibliography{root}

\end{document}
