\documentclass[11pt,a4paper]{article}
\usepackage{isabelle,isabellesym}

% further packages required for unusual symbols (see also
% isabellesym.sty), use only when needed

%\usepackage{amssymb}
  %for \<leadsto>, \<box>, \<diamond>, \<sqsupset>, \<mho>, \<Join>,
  %\<lhd>, \<lesssim>, \<greatersim>, \<lessapprox>, \<greaterapprox>,
  %\<triangleq>, \<yen>, \<lozenge>

%\usepackage[greek,english]{babel}
  %option greek for \<euro>
  %option english (default language) for \<guillemotleft>, \<guillemotright>

%\usepackage[latin1]{inputenc}
  %for \<onesuperior>, \<onequarter>, \<twosuperior>, \<onehalf>,
  %\<threesuperior>, \<threequarters>, \<degree>

%\usepackage[only,bigsqcap]{stmaryrd}
  %for \<Sqinter>

%\usepackage{eufrak}
  %for \<AA> ... \<ZZ>, \<aa> ... \<zz> (also included in amssymb)

%\usepackage{textcomp}
  %for \<cent>, \<currency>

% this should be the last package used
\usepackage{pdfsetup}

% urls in roman style, theory text in math-similar italics
\urlstyle{rm}
\isabellestyle{it}


\begin{document}

\title{Normalization by Evaluation}
\author{Klaus Aehlig and Tobias Nipkow}
\maketitle

\begin{abstract}
This article formalizes normalization by evaluation as implemented in
Isabelle. Lambda calculus plus term rewriting is compiled into a functional
program with pattern matching. It is proved that the result of a successful
evaluation is a) correct, i.e.\ equivalent to the input, and b) in normal
form.
\end{abstract}

An earlier version of this theory is described in a paper by Aehlig \emph{et
al.}~\cite{AehligHN-TPHOLs08}. The normal form proof is not in that paper.

\input{session}

\bibliographystyle{plain}
\bibliography{root}

\end{document}
