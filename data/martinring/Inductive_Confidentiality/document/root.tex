\documentclass[11pt,a4paper]{article}
\usepackage{isabelle,isabellesym}

% further packages required for unusual symbols (see also
% isabellesym.sty), use only when needed

%\usepackage{amssymb}
  %for \<leadsto>, \<box>, \<diamond>, \<sqsupset>, \<mho>, \<Join>,
  %\<lhd>, \<lesssim>, \<greatersim>, \<lessapprox>, \<greaterapprox>,
  %\<triangleq>, \<yen>, \<lozenge>

%\usepackage[greek,english]{babel}
  %option greek for \<euro>
  %option english (default language) for \<guillemotleft>, \<guillemotright>

%\usepackage[only,bigsqcap]{stmaryrd}
  %for \<Sqinter>

%\usepackage{eufrak}
  %for \<AA> ... \<ZZ>, \<aa> ... \<zz> (also included in amssymb)

%\usepackage{textcomp}
  %for \<onequarter>, \<onehalf>, \<threequarters>, \<degree>, \<cent>,
  %\<currency>

% this should be the last package used
\usepackage{pdfsetup}

% urls in roman style, theory text in math-similar italics
\urlstyle{rm}
\isabellestyle{it}

% for uniform font size
%\renewcommand{\isastyle}{\isastyleminor}


\begin{document}

\title{Inductive Study of Confidentiality}
\author{Giampaolo Bella\\
Dipartimento di Matematica e Informatica, Universit\`a di Catania, Italy}
\maketitle

\begin{abstract}
This document contains the full theory files accompanying article ``Inductive Study of Confidentiality --- for Everyone'' \cite{confeveryone}. They aim at an illustrative and didactic presentation of the Inductive Method of protocol analysis, focusing on the treatment of one of the main goals of security protocols: confidentiality against a threat model. The treatment of confidentiality, which in fact forms a key aspect of all protocol analysis tools, has been found cryptic by many learners of the Inductive Method, hence the motivation for this work. The theory files in this document guide the reader step by step towards design and proof of significant confidentiality theorems. These are developed against two threat models, the standard Dolev-Yao and a more audacious one, the General Attacker, which turns out to be particularly useful also for teaching purposes.
\end{abstract}

\tableofcontents

% sane default for proof documents
\parindent 0pt\parskip 0.5ex

% generated text of all theories
\input{session}

% optional bibliography
\bibliographystyle{abbrv}
\bibliography{root}

\end{document}

%%% Local Variables:
%%% mode: latex
%%% TeX-master: t
%%% End:
