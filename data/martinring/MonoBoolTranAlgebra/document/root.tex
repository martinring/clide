\documentclass[11pt,a4paper]{article}
\usepackage{isabelle,isabellesym}

% further packages required for unusual symbols (see also
% isabellesym.sty), use only when needed

%\usepackage{amssymb}
  %for \<leadsto>, \<box>, \<diamond>, \<sqsupset>, \<mho>, \<Join>,
  %\<lhd>, \<lesssim>, \<greatersim>, \<lessapprox>, \<greaterapprox>,
  %\<triangleq>, \<yen>, \<lozenge>

%\usepackage[greek,english]{babel}
  %option greek for \<euro>
  %option english (default language) for \<guillemotleft>, \<guillemotright>

\usepackage[only,bigsqcap]{stmaryrd}
  %for \<Sqinter>

%\usepackage{eufrak}
  %for \<AA> ... \<ZZ>, \<aa> ... \<zz> (also included in amssymb)

%\usepackage{textcomp}
  %for \<onequarter>, \<onehalf>, \<threequarters>, \<degree>, \<cent>,
  %\<currency>

% this should be the last package used
\usepackage{pdfsetup}

% urls in roman style, theory text in math-similar italics
\urlstyle{rm}
\isabellestyle{it}

% for uniform font size
%\renewcommand{\isastyle}{\isastyleminor}


\begin{document}

\title{Algebra of Monotonic Boolean Transformers}
\author{Viorel Preoteasa}
\maketitle
\begin{abstract}
Algebras of imperative programming languages have 
been successful in reasoning about programs. In general an 
algebra of programs is an algebraic structure with programs 
as elements and with program compositions (sequential composition, 
choice, skip) as algebra operations. Various versions of these 
algebras were introduced to model partial correctness, total 
correctness, refinement, demonic choice, and other aspects. 
We formalize here an algebra which can be used to model total
correctness, refinement, demonic and angelic choice. The basic 
model of this algebra are monotonic Boolean transformers (monotonic 
functions from a Boolean algebra to itself).
\end{abstract}

\tableofcontents

\section{Introduction}

Abstract algebra is a useful tool in mathematics. Rather than working with 
specific models like natural numbers and algebra of truth values, one could 
reason in a more abstract setting and obtain results which are more general 
and applicable in different models. Algebras of logics are very important 
tools in studying various aspects of logical systems. Algebras of programming 
theories have also a significant contribution to the simplification of 
reasoning about programs. Programs are elements of an algebra and program 
compositions and program constants (sequential composition, choice, iteration, 
skip, fail) are the operations of the algebra. These operations satisfy a number 
of relations which are used for reasoning about programs. Kleene algebra with 
tests (KAT) \cite{kozen:1997} is an extension of Kleene algebra and it is suitable 
for reasoning about programs in a partial correctness framework. Various 
versions of Kleene algebras have been introduced, ranging from Kleene algebra 
with domain \cite{desharnais:moller:struth:2006} and concurrent Kleene algebra 
\cite{hoare:moller:struth:wehrman:2009} to an algebra for separation logic 
\cite{dang:hofner:moller:2011}. 

Refinement Calculus \cite{back-1978,back-1980,back-wright-98,morgan-90} is a 
calculus based on (monotonic) predicate transformers suitable for program 
development in a total correctness framework. Within this calculus various 
aspects of imperative programming languages can be formalized. These include 
total correctness, partial correctness, demonic choice, and angelic choice. 
Demonic refinement algebra (DRA) was introduced in 
\cite{vonwright:2002,vonwright:2004} as a variation of KAT to allow also reasoning 
about total correctness. The intended model of DRA is the set of conjunctive 
predicate transformers and this algebra cannot represent angelic choice. 
General refinement algebra (GRA) was also introduced in \cite{vonwright:2004}, 
but few results were proved and they were mostly related to iteration. 
Although the intended model for GRA is the set of monotonic predicate 
transformers, GRA does not include the angelic choice operator. GRA has 
been further extended in \cite{solin:vonwright:2009} with enabledness and 
termination operators, and it was extended for probabilistic programs in 
\cite{meinicke:solin:2010}.

This formalization is based on \cite{preoteasa:2011c} where a different
extension of GRA is introduced. In \cite{preoteasa:2011c}  GRA 
is extended with a dual operator 
\cite{guerreiro:82,back:vonwright:1989,back:vonwright:1990,back-wright-98}. 
The intended model for this algebra is the set of monotonic Boolean transformers 
(monotonic functions from a Boolean algebra to itself). 

This formalization is structured as follows. Section 2 introduces the monotonic 
Boolean transformers that are the basic model of the algebra. Section 3 introduces 
the monotonic Boolean transformers algebra and some of its properties. 
Section 4 introduces the Boolean algebra of assertions. In section
5 we introduce standard program statements and we prove their Hoare 
total correctness rules.


% sane default for proof documents
\parindent 0pt\parskip 0.5ex

% generated text of all theories
\input{session}

% optional bibliography
\bibliographystyle{abbrv}
\bibliography{root}

\end{document}

%%% Local Variables:
%%% mode: latex
%%% TeX-master: t
%%% End:
