\documentclass[11pt,a4paper]{article}
\usepackage{isabelle,isabellesym}

% further packages required for unusual symbols (see also
% isabellesym.sty), use only when needed

%\usepackage{amssymb}
  %for \<leadsto>, \<box>, \<diamond>, \<sqsupset>, \<mho>, \<Join>,
  %\<lhd>, \<lesssim>, \<greatersim>, \<lessapprox>, \<greaterapprox>,
  %\<triangleq>, \<yen>, \<lozenge>

%\usepackage[greek,english]{babel}
  %option greek for \<euro>
  %option english (default language) for \<guillemotleft>, \<guillemotright>

%\usepackage[latin1]{inputenc}
  %for \<onesuperior>, \<onequarter>, \<twosuperior>, \<onehalf>,
  %\<threesuperior>, \<threequarters>, \<degree>

%\usepackage[only,bigsqcap]{stmaryrd}
  %for \<Sqinter>

%\usepackage{eufrak}
  %for \<AA> ... \<ZZ>, \<aa> ... \<zz> (also included in amssymb)

%\usepackage{textcomp}
  %for \<cent>, \<currency>

% this should be the last package used
\usepackage{pdfsetup}

% urls in roman style, theory text in math-similar italics
\urlstyle{rm}
\isabellestyle{it}

% for uniform font size
%\renewcommand{\isastyle}{\isastyleminor}


\begin{document}

\title{Formalizing Statecharts using Hierarchical Automata}
\author{Steffen Helke and Florian Kamm\"uller}
\maketitle

\begin{abstract}
We formalize in Isabelle/HOL the abtract syntax and a synchronous step
semantics for the specification language Statecharts \cite{HN96}. The
formalization is based on Hierarchical Automata \cite{MLS97} 
which allow a structural decomposition of Statecharts into Sequential
Automata. To support the composition of Statecharts, we introduce 
calculating operators to construct a Hierarchical Automaton in a 
stepwise manner \cite{HK01}. Furthermore, we present a complete
semantics of Statecharts including a theory of data spaces, which enables 
the modelling of racing effects \cite{HK05}. We also adapt CTL for 
Statecharts to build a bridge for future combinations with
model checking. However the main motivation of this work is to provide a 
sound and complete basis for reasoning on
Statecharts. As a central meta theorem we prove that the 
well-formedness of a Statechart is preserved by the 
semantics \cite{Hel07}.    
\end{abstract}

\tableofcontents

% sane default for proof documents
\parindent 0pt\parskip 0.5ex

% generated text of all theories
\input{session}

% optional bibliography
\bibliographystyle{abbrv}
%\bibliography{root}
\begin{thebibliography}{MLSH99}

\bibitem[HN96]{HN96}
D.~Harel and D.~Naamad.
\newblock {A STATEMATE semantics for statecharts}.
\newblock {\em ACM Transactions on Software Engineering and Methodology},
  5(4):293--333, Oct 1996.

\bibitem[MLS97]{MLS97}
E.~Mikk, Y.~Lakhnech, and M.~Siegel.
\newblock {Hierarchical automata as model for statecharts}.
\newblock In {\em Asian Computing Science Conference (ASIAN'97)},
\newblock \textit{Springer LNCS}, \textbf{1345}, 1997.
 
\bibitem[HK01]{HK01}
S.~Helke and F.~Kamm{\"u}ller.
\newblock {Representing Hierarchical Automata in Interactive Theorem Provers}.
\newblock In R. J. Boulton, P. B. Jackson, editors, {\em Theorem
Proving in Higher Order Logics, TPHOLs 2001}, \textit{Springer LNCS},
\textbf{2152}, 2001.

\bibitem[HK05]{HK05}
S.~Helke and F.~Kamm{\"u}ller.
\newblock {Structure Preserving Data Abstractions for Statecharts}.
\newblock In F. Wang, editors, {\em  Formal Techniques
  for Networked and Distributed Systems, FORTE 2005}, \textit{Springer LNCS},
\textbf{3731}, 2005.

\bibitem[Hel07]{Hel07}
S.~Helke.
\newblock {\em Verification of Statecharts using Structure- and
  Property-Preserving Data Abstraction {$[$}german{$]$} }.
\newblock PhD thesis, Fakult{\"a}t IV, Technische Universit{\"a}t Berlin, Germany, 2007.

\end{thebibliography}


\end{document}

%%% Local Variables:
%%% mode: latex
%%% TeX-master: t
%%% End:
