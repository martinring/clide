\documentclass[11pt,a4paper]{article}
\usepackage{isabelle,isabellesym}

% further packages required for unusual symbols (see also isabellesym.sty)
% use only when needed
\usepackage{a4wide}
\usepackage{amssymb}                  % for \<leadsto>, \<box>, \<diamond>,
                                       % \<sqsupset>, \<mho>, \<Join>, 
                                       % \<lhd>, \<lesssim>, \<greatersim>,
                                       % \<lessapprox>, \<greaterapprox>,
                                       % \<triangleq>, \<yen>, \<lozenge>
\usepackage[greek,english]{babel}      % greek for \<euro>,
                                       % english for \<guillemotleft>, 
                                       %             \<guillemotright>
                                       % default language = last
\usepackage[utf8]{inputenc}
\usepackage[only,bigsqcap]{stmaryrd}  % for \<Sqinter>
\usepackage{eufrak}                   % for \<AA> ... \<ZZ>, \<aa> ... \<zz>
                                       % (only needed if amssymb not used)
%\usepackage{textcomp}                 % for \<cent>, \<currency>

% this should be the last package used
\usepackage{pdfsetup}

% urls in roman style, theory text in math-similar italics
\urlstyle{rm}
\isabellestyle{tt}

\renewcommand{\isamarkupheader}[1]
{\newpage\markright{Theory~\isabellecontext}\section{#1}}
\renewcommand{\isamarkupsection}[1]{\subsection{#1}}
\renewcommand{\isamarkupsubsection}[1]{\subsubsection{#1}}

\begin{document}

\title{A Fast SAT Solver for Isabelle in Standard ML}
\author{Armin Heller}
\maketitle

\begin{abstract}
This contribution contains a fast SAT solver for Isabelle written in
Standard ML.  By loading the theory \isa{DPT\_SAT\_Solver}, the SAT solver
installs itself (under the name ``dptsat'') and certain Isabelle tools like
Refute will start using it automatically.
This is a port of the DPT (Decision Procedure Toolkit) SAT Solver written in
OCaml.

Theory \isa{DPT\_SAT\_Tests} tests the solver on a few hundred problems.
\end{abstract}

\tableofcontents

\parindent 0pt\parskip 0.5ex

% include generated text of all theories
\input{session}

\bibliographystyle{abbrv}
\bibliography{root}

\end{document}
