\documentclass[11pt,a4paper]{article}
\usepackage{isabelle,isabellesym}

% further packages required for unusual symbols (see also isabellesym.sty)
% use only when needed
\usepackage{amssymb}                  % for \<leadsto>, \<box>, \<diamond>,
                                       % \<sqsupset>, \<mho>, \<Join>, 
                                       % \<lhd>, \<lesssim>, \<greatersim>,
                                       % \<lessapprox>, \<greaterapprox>,
                                       % \<triangleq>, \<yen>, \<lozenge>
%\usepackage[greek,english]{babel}     % greek for \<euro>,
                                       % english for \<guillemotleft>, 
                                       %             \<guillemotright>
                                       % default language = last
%\usepackage[latin1]{inputenc}         % for \<onesuperior>, \<onequarter>,
                                       % \<twosuperior>, \<onehalf>,
                                       % \<threesuperior>, \<threequarters>,
                                       % \<degree>
%\usepackage[only,bigsqcap]{stmaryrd}  % for \<Sqinter>
%\usepackage{eufrak}                   % for \<AA> ... \<ZZ>, \<aa> ... \<zz>
                                       % (only needed if amssymb not used)
%\usepackage{textcomp}                 % for \<cent>, \<currency>

% this should be the last package used
\usepackage{pdfsetup}

% urls in roman style, theory text in math-similar italics
\urlstyle{rm}
\isabellestyle{it}

\newcommand\isafor{\textsf{IsaFoR}}
\newcommand\ceta{\textsf{Ce\kern-.18emT\kern-.18emA}}

\begin{document}

\title{Executable multivariate polynomials}
\author{Christian Sternagel and Ren\'e Thiemann}
\maketitle

\begin{abstract}
  We define multivariate polynomials over arbitrary (ordered)
  semirings in combination with (executable) operations like addition, multiplication,
  and substitution. We also define (weak) monotonicity of polynomials
  and comparison of polynomials where we provide standard estimations 
  like absolute positiveness or the more recent
  approach of \cite{NZM10}. Moreover, it is proven
  that strongly normalizing (monotone) orders 
  can be lifted to strongly normalizing (monotone) orders over polynomials. 
   
  Our formalization was performed as part of the \isafor/\ceta-system 
  \cite{CeTA}\footnote{\url{http://cl-informatik.uibk.ac.at/software/ceta}}
  which
  contains several termination techniques. The provided theories have been
  essential to formalize polynomial-interpretations \cite{L79,Rational}.
\end{abstract}

\tableofcontents


% include generated text of all theories
\input{session}



\bibliographystyle{abbrv}
\bibliography{root}

\end{document}
