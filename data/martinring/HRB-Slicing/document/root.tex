\documentclass[11pt,a4paper,notitlepage]{report}
\usepackage{isabelle,isabellesym}

% further packages required for unusual symbols (see also isabellesym.sty)
\usepackage{latexsym}
\usepackage{amssymb}
\usepackage{textcomp}
\usepackage[english]{babel}
\usepackage[utf8]{inputenc}
\usepackage{wasysym}
\usepackage{graphicx}

\usepackage{isabelle,isabellesym} 

\usepackage{url}

% this should be the last package used
\usepackage{pdfsetup}

% urls in roman style, theory text in math-similar italics
\urlstyle{rm}
\isabellestyle{it}

\newcommand{\isachapter}[1]
{\markright{\isabellecontext}\chapter{#1}}

\newcommand{\isaheader}[1]
{\markright{\isabellecontext}\section{#1}}
\renewcommand{\isamarkupheader}[1]{#1}
\renewcommand{\isamarkupsection}[1]{\subsection{#1}}


\begin{document}

\title{Backing up Slicing: Verifying the interprocedural two-phase Horwitz-Reps-Binkley Slicer}
\author{Daniel Wasserrab}
\maketitle

\begin{abstract}
Slicing is a widely-used technique with applications in e.g. compiler
technology and software security. Thus verification of
algorithms in these areas is often based on the correctness of slicing,
which should ideally be proven independent of concrete programming
languages and with the help of well-known verifying techniques such as
proof assistants. 

After verifying static intraprocedural and dynamic slicing \cite{Wasserrab:08}, we
focus now on the sophisticated interprocedural two-phase Horwitz-Reps-Binkley slicer
\cite{HorwitzRB:88}, including summary edges which were added in \cite{RepsHSR:94}.

Again, abstracting from concrete syntax we base our work on a graph
representation of the program fulfilling certain structural
and well-formedness properties. The framework is instantiated with a simple 
While language with procedures, showing its validity.
\end{abstract}

\parindent 0pt\parskip 0.5ex

% generated text of all theories
\input{session}

% optional bibliography
%\bibliographystyle{abbrv}
%\bibliography{root}
\begin{thebibliography}{10}

\bibitem{HorwitzRB:88}
Susan Horwitz and Thomas Reps and David Binkley.
\newblock Interprocedural Slicing Using Dependence Graphs.
\newblock {\em ACM Transactions on Programming Languages and Systems}, 12(1):26--60, 1990.

\bibitem{RepsHSR:94}
Thomas Reps and Susan Horwitz and Mooly Sagiv and Genevieve Rosay.
\newblock Speeding up slicing.
\newblock In {\em Proc. of FSE'94}, pages 11--20. ACM, 1994

\bibitem{Wasserrab:08}
\newblock Daniel Wasserrab.
\newblock Towards certified slicing.
\newblock In G. Klein, T. Nipkow, and L. Paulson, editors, {\em Archive of Formal
Proofs}.
\newblock \url{http://afp.sf.net/entries/Slicing.shtml}, September 2008.
\newblock Formal proof development.

\end{thebibliography}

\end{document}

%%% Local Variables:
%%% mode: latex
%%% TeX-master: t
%%% End:
