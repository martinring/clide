\documentclass[11pt,a4paper]{article}
\usepackage{isabelle,isabellesym}

% further packages required for unusual symbols (see also
% isabellesym.sty), use only when needed

%\usepackage{amssymb}
  %for \<leadsto>, \<box>, \<diamond>, \<sqsupset>, \<mho>, \<Join>,
  %\<lhd>, \<lesssim>, \<greatersim>, \<lessapprox>, \<greaterapprox>,
  %\<triangleq>, \<yen>, \<lozenge>

%\usepackage[greek,english]{babel}
  %option greek for \<euro>
  %option english (default language) for \<guillemotleft>, \<guillemotright>

%\usepackage[only,bigsqcap]{stmaryrd}
  %for \<Sqinter>

%\usepackage{eufrak}
  %for \<AA> ... \<ZZ>, \<aa> ... \<zz> (also included in amssymb)

%\usepackage{textcomp}
  %for \<onequarter>, \<onehalf>, \<threequarters>, \<degree>, \<cent>,
  %\<currency>

% this should be the last package used
\usepackage{pdfsetup}

% urls in roman style, theory text in math-similar italics
\urlstyle{rm}
\isabellestyle{it}

% for uniform font size
%\renewcommand{\isastyle}{\isastyleminor}


\begin{document}

\title{Well-Quasi-Orders}
\author{Christian Sternagel\thanks{%
  The research was funded by the Austrian Science Fund (FWF): J3202.}}
\maketitle

\begin{abstract}
Based on Isabelle/HOL's type class for preorders, we introduce a type class for
well-quasi-orders (wqo) which is characterized by the absence of ``bad''
sequences (our proofs are along the lines of the proof of Nash-Williams
\cite{N1963}, from which we also borrow terminology).  Our two main results are
instantiations for the product type and the list type, which (almost) directly
follow from our proofs of (1) Dickson's Lemma and (2) Higman's Lemma. More
concretely:
\begin{enumerate}
\item If the sets $A$ and $B$ are wqo then their Cartesian product is wqo.
\item If the set $A$ is wqo then the set of finite lists over $A$ is wqo.
\end{enumerate}
\end{abstract}

\tableofcontents


% sane default for proof documents
\parindent 0pt\parskip 0.5ex

% generated text of all theories
\input{session}

% optional bibliography
\bibliographystyle{abbrv}
\bibliography{root}

\end{document}

%%% Local Variables:
%%% mode: latex
%%% TeX-master: t
%%% End:
