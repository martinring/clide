\documentclass[11pt,a4paper]{article}
\usepackage{isabelle,isabellesym}

% further packages required for unusual symbols (see also
% isabellesym.sty), use only when needed

%\usepackage{amssymb}
  %for \<leadsto>, \<box>, \<diamond>, \<sqsupset>, \<mho>, \<Join>,
  %\<lhd>, \<lesssim>, \<greatersim>, \<lessapprox>, \<greaterapprox>,
  %\<triangleq>, \<yen>, \<lozenge>

%\usepackage[greek,english]{babel}
  %option greek for \<euro>
  %option english (default language) for \<guillemotleft>, \<guillemotright>

%\usepackage[latin1]{inputenc}
  %for \<onesuperior>, \<onequarter>, \<twosuperior>, \<onehalf>,
  %\<threesuperior>, \<threequarters>, \<degree>

%\usepackage[only,bigsqcap]{stmaryrd}
  %for \<Sqinter>

%\usepackage{eufrak}
  %for \<AA> ... \<ZZ>, \<aa> ... \<zz> (also included in amssymb)

%\usepackage{textcomp}
  %for \<cent>, \<currency>

% this should be the last package used
\usepackage{pdfsetup}

% urls in roman style, theory text in math-similar italics
\urlstyle{rm}
\isabellestyle{it}

% for uniform font size
%\renewcommand{\isastyle}{\isastyleminor}


\begin{document}

\title{Binomial Heaps and Skew Binomial Heaps}
\author{Rene Meis \and Finn Nielsen \and Peter Lammich}
\maketitle

\begin{abstract}
  We implement and prove correct binomial heaps and skew binomial heaps.
  Both are data-structures for priority queues.
  While binomial heaps have logarithmic {\em findMin}, {\em deleteMin}, 
  {\em insert}, and {\em meld }
  operations,
  skew binomial heaps have constant time {\em findMin}, {\em insert}, 
  and {\em meld} operations,
  and only the {\em deleteMin}-operation is logarithmic. This is achieved by
  using {\em skew links} to avoid cascading linking on {\em insert}-operations,
  and {\em data-structural bootstrapping} to get constant-time {\em findMin}
  and {\em meld} operations.
  Our implementation follows the paper of Brodal and Okasaki \cite{BrOk96}.
\end{abstract}

\clearpage

\tableofcontents

% sane default for proof documents
\parindent 0pt\parskip 0.5ex

% generated text of all theories
\input{session}

% optional bibliography
\bibliographystyle{abbrv}
\bibliography{root}

\end{document}

%%% Local Variables:
%%% mode: latex
%%% TeX-master: t
%%% End:
