\documentclass[11pt,a4paper]{article}
\usepackage{isabelle,isabellesym}

% further packages required for unusual symbols (see also
% isabellesym.sty), use only when needed

%\usepackage{amssymb}
  %for \<leadsto>, \<box>, \<diamond>, \<sqsupset>, \<mho>, \<Join>,
  %\<lhd>, \<lesssim>, \<greatersim>, \<lessapprox>, \<greaterapprox>,
  %\<triangleq>, \<yen>, \<lozenge>

%\usepackage[greek,english]{babel}
  %option greek for \<euro>
  %option english (default language) for \<guillemotleft>, \<guillemotright>

%\usepackage[latin1]{inputenc}
  %for \<onesuperior>, \<onequarter>, \<twosuperior>, \<onehalf>,
  %\<threesuperior>, \<threequarters>, \<degree>

%\usepackage[only,bigsqcap]{stmaryrd}
  %for \<Sqinter>

%\usepackage{eufrak}
  %for \<AA> ... \<ZZ>, \<aa> ... \<zz> (also included in amssymb)

%\usepackage{textcomp}
  %for \<cent>, \<currency>

% this should be the last package used
\usepackage{pdfsetup}

% urls in roman style, theory text in math-similar italics
\urlstyle{rm}
\isabellestyle{it}


\begin{document}

\title{Quantifier Elimination for Linear Arithmetic}
\author{Tobias Nipkow}
\maketitle

\begin{abstract}
This article formalizes quantifier elimination procedures for dense
linear orders, linear real arithmetic and Presburger arithmetic.  In
each case both a DNF-based non-elementary algorithm and one or more
(doubly) exponential NNF-based algorithms are formalized, including
the well-known algorithms by Ferrante and Rackoff and by Cooper. The
NNF-based algorithms for dense linear orders are new but based on
Ferrante and Rackoff and on an algorithm by Loos and Weisspfenning
which simulates infinitesimals.

All algorithms are directly executable. In particular, they yield
reflective quantifier elimination procedures for HOL itself.

The formalization makes heavy use of locales and is therefore highly modular.
\end{abstract}

\noindent
For an exposition of the DNF-based procedures see~\cite{Nipkow-MOD2007},
for the NNF-based procedures see~\cite{Nipkow-IJCAR08}.
\tableofcontents

% generated text of all theories
\input{session}

\bibliographystyle{plain}
\bibliography{root}

\end{document}
