\documentclass[11pt,a4paper]{article}
\usepackage{graphicx}
\usepackage{isabelle,isabellesym}

% further packages required for unusual symbols (see also isabellesym.sty)
% use only when needed
%\usepackage{amssymb}                  % for \<leadsto>, \<box>, \<diamond>,
                                       % \<sqsupset>, \<mho>, \<Join>, 
                                       % \<lhd>, \<lesssim>, \<greatersim>,
                                       % \<lessapprox>, \<greaterapprox>,
                                       % \<triangleq>, \<yen>, \<lozenge>
%\usepackage[greek,english]{babel}     % greek for \<euro>,
                                       % english for \<guillemotleft>, 
                                       %             \<guillemotright>
                                       % default language = last
%\usepackage[latin1]{inputenc}         % for \<onesuperior>, \<onequarter>,
                                       % \<twosuperior>, \<onehalf>,
                                       % \<threesuperior>, \<threequarters>,
                                       % \<degree>
%\usepackage[only,bigsqcap]{stmaryrd}  % for \<Sqinter>
%\usepackage{eufrak}                   % for \<AA> ... \<ZZ>, \<aa> ... \<zz>
                                       % (only needed if amssymb not used)
%\usepackage{textcomp}                 % for \<cent>, \<currency>

% this should be the last package used
\usepackage{pdfsetup}

% urls in roman style, theory text in math-similar italics
\urlstyle{rm}
\isabellestyle{it}


\begin{document}

\title{Countable Ordinals}
\author{Brian Huffman}
\maketitle
\begin{abstract}
This development defines a well-ordered type of countable ordinals.
It includes notions of continuous and normal functions, recursively
defined functions over ordinals, least fixed-points, and derivatives.
Much of ordinal arithmetic is formalized, including exponentials and
logarithms. The development concludes with formalizations of Cantor
Normal Form and Veblen hierarchies over normal functions.
\end{abstract}

\tableofcontents

%\begin{center}
%  \includegraphics[scale=0.7]{session_graph}
%\end{center}

\newpage

\parindent 0pt\parskip 0.5ex

% generated text of all theories
\input{session}

% optional bibliography
%\bibliographystyle{abbrv}
%\bibliography{root}

\end{document}

%%% Local Variables:
%%% mode: latex
%%% TeX-master: t
%%% End:
