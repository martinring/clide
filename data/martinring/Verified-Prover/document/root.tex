

\documentclass[11pt,a4paper]{article}
\usepackage{isabelle,isabellesym}

% further packages required for unusual symbols (see also isabellesym.sty)
% use only when needed
%\usepackage{amssymb}                  % for \<leadsto>, \<box>, \<diamond>,
                                       % \<sqsupset>, \<mho>, \<Join>, 
                                       % \<lhd>, \<lesssim>, \<greatersim>,
                                       % \<lessapprox>, \<greaterapprox>,
                                       % \<triangleq>, \<yen>, \<lozenge>
%\usepackage[greek,english]{babel}     % greek for \<euro>,
                                       % english for \<guillemotleft>, 
                                       %             \<guillemotright>
                                       % default language = last
%\usepackage[latin1]{inputenc}         % for \<onesuperior>, \<onequarter>,
                                       % \<twosuperior>, \<onehalf>,
                                       % \<threesuperior>, \<threequarters>,
                                       % \<degree>
%\usepackage[only,bigsqcap]{stmaryrd}  % for \<Sqinter>
%\usepackage{eufrak}                   % for \<AA> ... \<ZZ>, \<aa> ... \<zz>
                                       % (only needed if amssymb not used)
%\usepackage{textcomp}                 % for \<cent>, \<currency>

% this should be the last package used
\usepackage{pdfsetup}

% urls in roman style, theory text in math-similar italics
\urlstyle{rm}
\isabellestyle{it}


\begin{document}

\title{A Mechanically Verified, Efficient, Sound and Complete Theorem Prover For First Order Logic}
\author{Tom Ridge}
\maketitle

\begin{abstract}
  Building on work by Wainer and Wallen, formalised by James
  Margetson, we present soundness and completeness proofs for a system
  of first order logic. The completeness proofs naturally suggest an
  algorithm to derive proofs. This algorithm can be implemented in a
  tail recursive manner. We provide the formalisation in Isabelle/HOL\@.
  The algorithm can be executed via the rewriting tactics of Isabelle.
  Alternatively, we transport the definitions to OCaml, to give a
  directly executable program.
\end{abstract}

\tableofcontents

\parindent 0pt\parskip 0.5ex

\section{Introduction}

Wainer and Wallen gave soundness and completeness proofs for first
order logic in \cite{Wainer:92}. This material was later formalised by
James Margetson \cite{margetson99completeness}. We ported this to the
current version of Isabelle in \cite{margetson04completeness}. Drawing
on some of the proofs in previous versions, especially the proof of
soundness for the $\forall I$ rule, we formalise modified proofs, for
a related system. Implicit in \cite{Wainer:92}, and noted by Margetson
in \cite{margetson99completeness}, is that the proofs of completeness
suggest a constructive algorithm. We derive this algorithm, which
turns out to be tail recursive, and this is the origin of our claim
for efficiency. The algorithm can be executed in Isabelle using the
rewriting engine. Alternatively, we provide an implementation in
Ocaml.

\section{Formalisation}
% include generated text of all theories
\input{session}

\section{Optimisation and Extension}

There are plenty of obvious optimisations. The first medium level
optimisation is to avoid the recomputation of newvars by incorporating
the maxvar into a sequent. At a low level, most of the list operations
are just moving a pointer along a list: only FConj requires
duplicating a list. Reporting ``not provable'' on obviously non-provable
goals would be useful, as would a more efficient choice of witnessing
terms for existentials.

In terms of extensions, the obvious targets
are function terms and equality. 



\section{OCaml Implementation}

\begin{verbatim}
open List;;

type pred = int;;

type var = int;;

type form = 
    PAtom of (pred*(var list))
  | NAtom of (pred*(var list))
  | FConj of form * form
  | FDisj of form * form
  | FAll of form
  | FEx of form
;;

let rec preSuc t = match t with
    [] -> []
  | (a::list) -> (match a with 0 -> preSuc list | sucn -> (sucn-1::preSuc list));;

let rec fv t = match t with
    PAtom (p,vs) -> vs
  | NAtom (p,vs) -> vs
  | FConj (f,g) -> (fv f)@(fv g)
  | FDisj (f,g) -> (fv f)@(fv g)
  | FAll f -> preSuc (fv f)
  | FEx f -> preSuc (fv f);;

let suc x = x+1;;

let bump phi y = match y with 0 -> 0 | sucn -> suc (phi (sucn-1));;

let rec subst r f = match f with
    PAtom (p,vs) -> PAtom (p,map r vs)
  | NAtom (p,vs) -> NAtom (p,map r vs)
  | FConj (f,g) -> FConj (subst r f, subst r g)
  | FDisj (f,g) -> FDisj (subst r f, subst r g)
  | FAll f -> FAll (subst (bump r) f)
  | FEx f -> FEx (subst (bump r) f);;

let finst body w = subst (fun v -> match v with 0 -> w | sucn -> (sucn-1)) body;;

let s_of_ns ns = map snd ns;;

let sfv s = flatten (map fv s);;

let rec maxvar t = match t with
    [] -> 0
  | (a::list) -> max a (maxvar list);;

let newvar vs = suc (maxvar vs);;

let subs t = match t with 
    [] -> [[]]
  | (x::xs) -> let (m,f) = x in
      match f with 
          PAtom (p,vs) -> if mem (NAtom (p,vs)) (map snd xs) then [] else [xs@[(0,PAtom (p,vs))]]
        | NAtom (p,vs) -> if mem (PAtom (p,vs)) (map snd xs) then [] else [xs@[(0,NAtom (p,vs))]]
        | FConj (f,g)  -> [xs@[(0,f)];xs@[(0,g)]]
        | FDisj (f,g) -> [xs@[(0,f);(0,g)]]
        | FAll f -> [xs@[(0,finst f (newvar (sfv (s_of_ns (x::xs)))))]]
        | FEx f -> [xs@[(0,finst f m);(suc m,FEx f)]];;

let rec prove' l = (if l = [] then true else prove' ((fun x -> flatten (map subs x)) l));;

let prove s = prove' [s];;

let my_f = FDisj (
  (FAll (FConj ((NAtom (0,[0])), (NAtom (1,[0])))),
  (FDisj ((FEx ((PAtom (1,[0])))),(FEx (PAtom (0,[0])))))));;

prove [(0,my_f)];;
\end{verbatim}

\bibliographystyle{abbrv}
\bibliography{root}

\end{document}
