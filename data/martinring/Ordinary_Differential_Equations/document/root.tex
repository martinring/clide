\documentclass[11pt,a4paper]{article}
\usepackage[utf8]{inputenc}
\usepackage{isabelle,isabellesym}

% further packages required for unusual symbols (see also
% isabellesym.sty), use only when needed

\usepackage{amssymb}
  %for \<leadsto>, \<box>, \<diamond>, \<sqsupset>, \<mho>, \<Join>,
  %\<lhd>, \<lesssim>, \<greatersim>, \<lessapprox>, \<greaterapprox>,
  %\<triangleq>, \<yen>, \<lozenge>

\usepackage[greek,english]{babel}
  %option greek for \<euro>
  %option english (default language) for \<guillemotleft>, \<guillemotright>

%\usepackage[only,bigsqcap]{stmaryrd}
  %for \<Sqinter>

%\usepackage{eufrak}
  %for \<AA> ... \<ZZ>, \<aa> ... \<zz> (also included in amssymb)

%\usepackage{textcomp}
  %for \<onequarter>, \<onehalf>, \<threequarters>, \<degree>, \<cent>,
  %\<currency>

% this should be the last package used
\usepackage{pdfsetup}

% urls in roman style, theory text in math-similar italics
\urlstyle{rm}
\isabellestyle{it}

% for uniform font size
%\renewcommand{\isastyle}{\isastyleminor}

\newcommand{\Nat}{\mathbb{N}}
\newcommand{\Real}{\mathbb{R}}
\newcommand{\Float}{\mathbb{F}}
\newcommand{\Bool}{\mathbb{B}}
\newcommand{\Eucl}[1]{\mathbb{R}^{#1}}
\newcommand{\UNIV}{\mathcal{U}}
\newcommand{\bcontfun}{\ensuremath{\mathcal{\overline{C}}}}
\newcommand{\grid}{\ensuremath{\Delta}}
\newcommand{\Kdo}{\textit{\textbf{do}}}
\newcommand{\Kif}{\textit{\textbf{if}}}
\newcommand{\Kthen}{\textit{\textbf{then}}}
\newcommand{\Kelse}{\textit{\textbf{else}}}

\newcommand{\Cfloat}[1]{\ensuremath{(#1)_\Float}}
\newcommand{\Creal}[1]{\ensuremath{(#1)_\Real}}

\newcommand{\keyword}[1]{\ensuremath{\textsl{\textsf{#1}}}}
\newcommand{\bkeyword}[1]{\ensuremath{\textsl{\textsf{\textbf{#1}}}}}

\newcommand{\Kunique}[1]{\ensuremath{\keyword{#1}}}

\newcommand{\Kabsbc}{\ensuremath{\keyword{Abs}_{\;\bcontfun}}}
\newcommand{\Kboundedderiv}{\keyword{deriv-bnd}}
\newcommand{\Kbound}{\keyword{bound}}
\newcommand{\Kclamp}{\keyword{clamp}}
\newcommand{\Kcomplete}{\keyword{complete}}
\newcommand{\Kconsistent}{\keyword{consistent}}
\newcommand{\Kconvergent}{\keyword{convergent}}
\newcommand{\Kcontinuous}{\keyword{continuous}}
\newcommand{\Kcontinue}{\keyword{continue}}
\newcommand{\KE}{\keyword{E}}
\newcommand{\Keuler}{\keyword{euler}}
\newcommand{\Keulerf}{\ensuremath{\widetilde{\Keuler}}{}}
\newcommand{\Keulerrect}{\keyword{euler-on-rectangle}}
\newcommand{\Keulerround}{\keyword{euler-rounded}}
\newcommand{\KFloat}{\keyword{Float}}
\newcommand{\Kgridfunction}{\keyword{gf}}
\newcommand{\Khassolution}{\keyword{has-solution}}
\newcommand{\Khmax}{\ensuremath{h_{max}}}
\newcommand{\Khmin}{\ensuremath{h_{min}}}
\newcommand{\Kinf}{\keyword{inf}}
\newcommand{\Kisinterval}{\keyword{is-interval}}
\newcommand{\Kissolution}{\keyword{is-solution}}
\newcommand{\Kivp}{\keyword{ivp}}
\newcommand{\Klipschitz}{\keyword{lipschitz}}
\newcommand{\Klipschitzl}{\keyword{local-lipschitz}}
\newcommand{\Kmaxstep}{\keyword{step-bnd}}
\newcommand{\Kopen}{\keyword{open}}
\newcommand{\KP}{\keyword{P}}
\newcommand{\KPb}{\ensuremath{\overline{\keyword{P}}}}
\newcommand{\Kpsieuler}{\ensuremath{\psi_{\mathsf{euler}}}}
\newcommand{\Kpsifeuler}{\ensuremath{\overline{\psi_{\mathsf{euler}}}}}
\newcommand{\Krepbc}{\ensuremath{\keyword{Rep}_{\,\bcontfun}}}
\newcommand{\Kround}{\keyword{round}}
\newcommand{\Krounded}{\keyword{rounded}}
\newcommand{\Ksolution}{\keyword{solution}}
\newcommand{\Kstable}{\keyword{stable}}
\newcommand{\Ksup}{\keyword{sup}}
\newcommand{\Kusolution}{\keyword{unique-solution}}
\newcommand{\Kubinterval}{\Kunique{bnd-strip}}
\newcommand{\Kuinterval}{\Kunique{strip}}
\newcommand{\Kurectangle}{\Kunique{rect}}
\newcommand{\Kuopen}{\Kunique{open-domain}}

\title{Numerical Analysis of Ordinary Differential Equations}
\author{Fabian Immler and Johannes Hölzl}

\begin{document}

\maketitle

\begin{abstract}

  Since many ordinary differential equations (ODEs) do not have a
  closed solution, approximating them is an important problem in
  numerical analysis. This work formalizes a method to approximate
  solutions of ODEs in Isabelle/HOL.

  We formalize initial value problems (IVPs) of ODEs and prove the
  existence of a unique solution, i.e.\ the Picard-Lindelöf
  theorem. We introduce general one-step methods for numerical
  approximation of the solution and provide an analysis regarding the
  local and global error of one-step methods.

  We give an executable specification of the Euler method to
  approximate the solution of IVPs. With user-supplied proofs for
  bounds of the differential equation we can prove an explicit bound
  for the global error. We use arbitrary-precision floating-point
  numbers and also handle rounding errors when we truncate the numbers
  for efficiency reasons.

\end{abstract}

\section{Relations to the paper}

Our paper~\cite{immlerhoelzl} is structured
roughly according to the sources you find here. In the following list we
show which notions of the paper correspond to which parts of the source code:

\begin{itemize}
\item Arbitrary Precision Floats ($\KFloat$): Included in the Library of the
  Isabelle/HOL distribution and in section `Floating-Point Numbers'
\item Bounded Continuous Functions ($\bcontfun$): Typedef \textit{bcontfun}
  in Section \ref{sec:bcontfun}
\item Initial Value Problems in Section \ref{sec:solutions}:
  \begin{itemize}
  \item IVP $\Kivp$: Definition as locale \textit{ivp}
  \item $\Kissolution$, $\Ksolution$, $\Kusolution$: Definition
    \textit{is-solution}, Definition \textit{solution}, Locale
    \textit{unique-solution}
  \item Combining solutions in Section \ref{sec:combining-solutions}
  \end{itemize}
\item Quantitative Picard-Lindeloef:
  \begin{itemize}
  \item Lipschitz continuity $\Klipschitz$: Definition
    \textit{lipschitz} in Section \ref{sec:lipschitz}
  \item Unique solution on bounded interval $\Kubinterval$: Locale
    \textit{unique-on-bounded-strip} in Section \ref{sec:pl-bi}
  \item Theorem 1 (Picard-Lindeloef): Sublocale
    \textit{unique-on-bounded-strip} $\subseteq$
    \textit{unique-solution} in Section \ref{sec:ivp-ubs}
  \item Unique solution on arbitrary interval $\Kuinterval$: Locale
    \textit{unique-on-strip} in Section \ref{sec:pl-us}
  \item Unique solution on rectangular domain $\Kurectangle$: Locale
    \textit{unique-on-rectangle} in Section \ref{sec:pl-rect}
  \item Theorem 2 (Picard-Lindeloef on a restricted domain): Sublocale
    \textit{unique-on-rectangle} $\subseteq$
    \textit{unique-solution} in Section \ref{sec:pl-rect}
 \end{itemize}
\item Qualitative Picard-Lindeloef: Section \ref{sec:qpl},
 \begin{itemize}
  \item Local Lipschitz continuity $\Klipschitzl$: Definition
    \textit{local-lipschitz} in Section \ref{sec:qpl-lipschitz}
  \item Open domain $\Kuopen$: Locale \textit{unique-on-open} in
    Section \ref{sec:qpl-lipschitz}
  \item Set of solutions $\Phi$: Definition \textit{PHI} in Section \ref{sec:qpl-global-solution}
  \item Theorem 3 (Maximal existence interval): Lemma \textit{global-solution} in
    Section \ref{sec:qpl-global-solution}
  \end{itemize}
\item One-step methods: Section \ref{sec:osm}
  \begin{itemize}
  \item Grid $\Delta$: Locale \textit{grid} in Section \ref{sec:osm-grid}
  \item Discrete Evolution $\Psi$: Definition
    \textit{discrete-evolution} in Section \ref{sec:osm-definition}
  \item Grid-function $\Kgridfunction$: Definition
    \textit{grid-function} in Section \ref{sec:osm-definition}
  \item Consistency $\Kconsistent$: Definition \textit{consistent} in
    Section \ref{sec:osm-consistent}
  \item Assumptions for convergence $\Kconvergent + \Kmaxstep$:
    Locales \textit{consistent-one-step},
    \textit{convergent-one-step}, \textit{max-step} in Section
    \ref{sec:osm-cons-imp-conv}
  \item Theorem 4 (Convergence of One-Step methods): Lemma
    \textit{convergence} in Section \ref{sec:osm-cons-imp-conv}
  \item Stability $\Kstable$: Locale \textit{stable-one-step} in
    Section \ref{sec:osm-stability}
  \item Theorem 5 (Stability of One-Step methods): Lemma
    \textit{stability} in Section \ref{sec:osm-stability}
 \end{itemize}
\item Euler method: Section \ref{sec:rk}
  \begin{itemize}
  \item Definition $\Keuler^f$: Definition \textit{euler} in Section
    \ref{sec:rk-definition}
  \item $\Kboundedderiv$: Locale \textit{bounded-derivative} in
    Section \ref{sec:rk-euler-cons}
  \item Theorem 6 (Convergence of Euler): Lemma \textit{convergence} in
    Section \ref{sec:rk-euler-conv-on-rect}
  \item $\Keulerround$: Locale \textit{euler-rounded-on-rectangle} in
    Section \ref{sec:rk-euler-stable}
  \item Theorem 7 (Convergence of the approximate Euler method on
    $\Float$): Lemma \textit{convergence-qfloat} in Section
    \ref{sec:rk-euler-stable}
  \end{itemize}
\item Example: Section \ref{sec:example}
\end{itemize}

\tableofcontents

% sane default for proof documents
\parindent 0pt\parskip 0.5ex

% generated text of all theories
\input{session}

% optional bibliography
\bibliographystyle{abbrv}
\bibliography{root}

\end{document}

%%% Local Variables:
%%% mode: latex
%%% TeX-master: t
%%% End:
