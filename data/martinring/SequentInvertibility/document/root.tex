\documentclass{llncs}
\usepackage{isabelle,isabellesym}
\usepackage{latexsym,proof,stmaryrd}
\usepackage[english]{babel}
% this should be the last package used
\usepackage{pdfsetup}

% urls in roman style, theory text in math-similar italics
\urlstyle{rm}
\isabellestyle{it}

% Linenumber gubbins



\newcommand{\implies}[2]{#1 \!\supset\! #2}       
\newcommand{\SSeq}[3]{                  % judgement with stoup
\mbox{$#1\raisebox{.2mm}{$\,\,\stackrel{{}^{#2}}{\Longrightarrow}\,\,$}#3$}}
\newcommand{\SC}{uniprincipal }
\newcommand{\SCCap}{Uniprincipal }
\newcommand{\eat}[1]{}
\newcommand{\com}{combinable }
\newcommand{\Com}{Combinable }
\newcommand{\commed}{combined }
\newcommand{\Commed}{Combined }
\newcommand{\isasymLM}{\isamath{\Lbag}}
\newcommand{\isasymRM}{\isamath{\Rbag}}
\newcommand{\isasymEmpt}{\isamath{\emptyset}}
% Make the comments within proofs the same size as elsewhere
\renewcommand{\isastyletxt}{\isastyletext}
% for uniform font size
%\renewcommand{\isastyle}{\isastyleminor}
\date{}
\title{Invertibility in Sequent Calculi}
\author{Peter Chapman}
\institute{School of Computer Science, University of St Andrews \\
Email: \texttt{pc@cs.st-andrews.ac.uk}}
\begin{document}

\maketitle


\begin{abstract}
The invertibility of the rules of a sequent calculus is important for guiding proof search and can be used in some formalised proofs of Cut admissibility.
We present  sufficient conditions for when a rule is invertible with respect to a calculus.  We illustrate the conditions with  examples.   It must be noted we give purely syntactic criteria; no guarantees are given as to the suitability of the rules.
\end{abstract}
% sane default for proof documents
% \parindent 0pt\parskip 0.5ex

% generated text of all theories
\input{session}

% optional bibliography
\bibliographystyle{plain}
\bibliography{root}

\end{document}

%%% Local Variables:
%%% mode: latex
%%% TeX-master: t
%%% End:
