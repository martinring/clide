

\documentclass[11pt,a4paper]{article}
\usepackage{isabelle,isabellesym}

% further packages required for unusual symbols (see also isabellesym.sty)
% use only when needed
%\usepackage{amssymb}                  % for \<leadsto>, \<box>, \<diamond>,
                                       % \<sqsupset>, \<mho>, \<Join>, 
                                       % \<lhd>, \<lesssim>, \<greatersim>,
                                       % \<lessapprox>, \<greaterapprox>,
                                       % \<triangleq>, \<yen>, \<lozenge>
%\usepackage[greek,english]{babel}     % greek for \<euro>,
                                       % english for \<guillemotleft>, 
                                       %             \<guillemotright>
                                       % default language = last
%\usepackage[latin1]{inputenc}         % for \<onesuperior>, \<onequarter>,
                                       % \<twosuperior>, \<onehalf>,
                                       % \<threesuperior>, \<threequarters>,
                                       % \<degree>
%\usepackage[only,bigsqcap]{stmaryrd}  % for \<Sqinter>
%\usepackage{eufrak}                   % for \<AA> ... \<ZZ>, \<aa> ... \<zz>
                                       % (only needed if amssymb not used)
%\usepackage{textcomp}                 % for \<cent>, \<currency>

% this should be the last package used
\usepackage{pdfsetup}

% urls in roman style, theory text in math-similar italics
\urlstyle{rm}
\isabellestyle{it}


\begin{document}

\title{RSAPSS}
\author{Christina Lindenberg and Kai Wirt \\ Darmstadt Technical University \\ Cryptography and Computeralgebra}
\maketitle

\begin{abstract}
  Formal verification is getting more and more important in computer science.
  However the state of the art formal verification methods in cryptography are
  very rudimentary. These theories are one step to provide a tool box allowing 
  the use of formal methods in every aspect of cryptography. Moreover we present
  a proof of concept for the feasibility of verification techniques to
  a standard signature algorithm.
\end{abstract}

\tableofcontents

\parindent 0pt\parskip 0.5ex

% include generated text of theories

\input{WordOperations}
\input{SHA1Padding}
\input{SHA1}

\input{Crypt}
\input{Mod}
\input{Pdifference}
\input{Productdivides}
\input{Pigeonholeprinciple}
\input{Fermat}
\input{Cryptinverts}

\input{Wordarith}

\input{EMSAPSS}
\input{RSAPSS}

\nocite{Bellare-Rogaway:98PSS}
\nocite{ Boyer-Moore:82RSA}
\nocite{ Nipkow-Paulson-Wenzel:02Isabelle}
\nocite{ PKCS}
\nocite{ Rivest-Shamir-Adleman:78RSA}
\nocite{ TU-Munich:05Isabelle}
\nocite{ fips:02SHA}
\bibliographystyle{abbrv}
\bibliography{root}

\end{document}
