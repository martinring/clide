\documentclass[11pt,a4paper]{article}
\usepackage{isabelle,isabellesym}

% further packages required for unusual symbols (see also isabellesym.sty)
% use only when needed
%\usepackage{amssymb}                  % for \<leadsto>, \<box>, \<diamond>,
                                       % \<sqsupset>, \<mho>, \<Join>, 
                                       % \<lhd>, \<lesssim>, \<greatersim>,
                                       % \<lessapprox>, \<greaterapprox>,
                                       % \<triangleq>, \<yen>, \<lozenge>
%\usepackage[greek,english]{babel}     % greek for \<euro>,
                                       % english for \<guillemotleft>, 
                                       %             \<guillemotright>
                                       % default language = last
%\usepackage[latin1]{inputenc}         % for \<onesuperior>, \<onequarter>,
                                       % \<twosuperior>, \<onehalf>,
                                       % \<threesuperior>, \<threequarters>,
                                       % \<degree>
%\usepackage[only,bigsqcap]{stmaryrd}  % for \<Sqinter>
%\usepackage{eufrak}                   % for \<AA> ... \<ZZ>, \<aa> ... \<zz>
                                       % (only needed if amssymb not used)
%\usepackage{textcomp}                 % for \<cent>, \<currency>

% this should be the last package used
\usepackage{pdfsetup}

% urls in roman style, theory text in math-similar italics
\urlstyle{rm}
\isabellestyle{it}

\newcommand\isafor{\textsf{IsaFoR}}
\newcommand\ceta{\textsf{Ce\kern-.18emT\kern-.18emA}}

\begin{document}

\title{Executable Transitive Closures\footnote{Supported by FWF (Austrian Science Fund) project P22767-N13.}}
\author{Ren\'e Thiemann}
\maketitle

\begin{abstract}
  We provide a generic work-list algorithm to compute the (reflexi\-\mbox{ve-)}transitive closure of
  relations where only successors of newly detected states are generated.
  In contrast to our previous work \cite{rtrancl_fin}, the relations do not have to be finite, 
  but each element must only have finitely many (indirect) successors. 
  Moreover, a subsumption relation can be used instead of pure equality.
  An executable variant of the algorithm is available where the generic operations are instantiated
  with list operations.
    
    This formalization was performed as part of the \isafor/\ceta{} project%
  \footnote{\url{http://cl-informatik.uibk.ac.at/software/ceta}} \cite{CeTA},
  and it has been used to certify size-change 
  termination proofs where large transitive closures have to be computed.
\end{abstract}

\tableofcontents


% include generated text of all theories
\input{session}



\bibliographystyle{abbrv}
\bibliography{root}

\end{document}
