\documentclass[11pt,a4paper]{article}
\usepackage{isabelle,isabellesym}

% further packages required for unusual symbols (see also
% isabellesym.sty), use only when needed

%\usepackage{amssymb}
  %for \<leadsto>, \<box>, \<diamond>, \<sqsupset>, \<mho>, \<Join>,
  %\<lhd>, \<lesssim>, \<greatersim>, \<lessapprox>, \<greaterapprox>,
  %\<triangleq>, \<yen>, \<lozenge>

%\usepackage[greek,english]{babel}
  %option greek for \<euro>
  %option english (default language) for \<guillemotleft>, \<guillemotright>

%\usepackage[only,bigsqcap]{stmaryrd}
  %for \<Sqinter>

%\usepackage{eufrak}
  %for \<AA> ... \<ZZ>, \<aa> ... \<zz> (also included in amssymb)

%\usepackage{textcomp}
  %for \<onequarter>, \<onehalf>, \<threequarters>, \<degree>, \<cent>,
  %\<currency>

% this should be the last package used
\usepackage{pdfsetup}

% urls in roman style, theory text in math-similar italics
\urlstyle{rm}
\isabellestyle{it}

% for uniform font size
%\renewcommand{\isastyle}{\isastyleminor}


\begin{document}

\title{Stuttering equivalence}
\author{
  Stephan Merz\\
  Inria Nancy \& LORIA\\
  Villers-l\`es-Nancy, France
}
\maketitle

\noindent%
Two $\omega$-sequences are stuttering equivalent if they differ only by
finite repetitions of elements. For example, the two sequences
\[
  (abbccca)^{\omega} \qquad\textrm{and}\qquad
  (aaaabc)^{\omega}
\]
are stuttering equivalent, whereas
\[
  (abac)^{\omega} \qquad\textrm{and}\qquad
  (aaaabcc)^{\omega}
\]
are not. Stuttering equivalence is a fundamental concept in the theory
of concurrent and distributed systems. Notably, Lamport~\cite{lamport:what-good}
argues that refinement notions for such systems should be insensitive to
finite stuttering. Peled and Wilke~\cite{peled:ltl-x} show that all LTL
(linear-time temporal logic) properties that are insensitive to stuttering
equivalence can be expressed without the next-time operator. Stuttering
equivalence is also important for certain verification techniques
such as partial-order reduction for model checking.

We formalize stuttering equivalence in Isabelle/HOL. Our development relies
on the notion of a stuttering sampling function that identifies blocks of 
identical sequence elements.


\tableofcontents

% sane default for proof documents
\parindent 0pt\parskip 0.5ex

% generated text of all theories
\input{session}

% optional bibliography
\bibliographystyle{abbrv}
\bibliography{root}

\end{document}

%%% Local Variables:
%%% mode: latex
%%% TeX-master: t
%%% End:
