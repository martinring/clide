\documentclass[11pt,a4paper]{article}
\usepackage{isabelle,isabellesym}

% further packages required for unusual symbols (see also
% isabellesym.sty), use only when needed

\usepackage{amssymb}
  %for \<leadsto>, \<box>, \<diamond>, \<sqsupset>, \<mho>, \<Join>,
  %\<lhd>, \<lesssim>, \<greatersim>, \<lessapprox>, \<greaterapprox>,
  %\<triangleq>, \<yen>, \<lozenge>

\usepackage[greek,english]{babel}
  %option greek for \<euro>
  %option english (default language) for \<guillemotleft>, \<guillemotright>

%\usepackage[latin1]{inputenc}
  %for \<onesuperior>, \<onequarter>, \<twosuperior>, \<onehalf>,
  %\<threesuperior>, \<threequarters>, \<degree>

\usepackage[only,bigsqcap]{stmaryrd}
  %for \<Sqinter>

%\usepackage{eufrak}
  %for \<AA> ... \<ZZ>, \<aa> ... \<zz> (also included in amssymb)

%\usepackage{textcomp}
  %for \<cent>, \<currency>

% this should be the last package used
\usepackage{pdfsetup}

% urls in roman style, theory text in math-similar italics
\urlstyle{rm}
\isabellestyle{it}

% for uniform font size
%\renewcommand{\isastyle}{\isastyleminor}

% Tweaks
\newcounter{TTStweak_tag}
\setcounter{TTStweak_tag}{0}
\newcommand{\setTTS}{\setcounter{TTStweak_tag}{1}}
\newcommand{\resetTTS}{\setcounter{TTStweak_tag}{0}}
\newcommand{\insertTTS}{\ifnum\value{TTStweak_tag}=1 \ \ \ \fi}

\renewcommand{\isakeyword}[1]{\resetTTS\emph{\bf\def\isachardot{.}\def\isacharunderscore{\isacharunderscorekeyword}\def\isacharbraceleft{\{}\def\isacharbraceright{\}}#1}}
\renewcommand{\isachardoublequoteopen}{\insertTTS}
\renewcommand{\isachardoublequoteclose}{\setTTS}
\renewcommand{\isanewline}{\mbox{}\par\mbox{}\resetTTS}

\renewcommand{\isamarkupcmt}[1]{\hangindent5ex{\isastylecmt --- #1}}


\begin{document}

\title{Tree Automata}
\author{Peter Lammich}
\maketitle

\begin{abstract}
  This work presents a machine-checked tree automata library for Standard-ML, OCaml and Haskell.
  The algorithms are efficient by using appropriate data structures like RB-trees.
  The available algorithms for non-deterministic automata include membership query, reduction, intersection,
  union, and emptiness check with computation of a witness for non-emptiness.
  
  The executable algorithms are derived from less-concrete, non-executable algorithms using
  data-refinement techniques. The concrete data structures are from the Isabelle Collections Framework.

  Moreover, this work contains a formalization of the class of tree-regular languages and its closure properties under set operations.
\end{abstract}

\clearpage

\tableofcontents

\clearpage

% sane default for proof documents
\parindent 0pt\parskip 0.5ex


\section{Introduction}


%Isabelle theory names
\newcommand{\Lang}{\mbox{\rm\textsf{\small Language$\_$Semantics}}}
\newcommand{\Duri}{\mbox{\rm\textsf{\small During$\_$Execution}}}
\newcommand{\Afte}{\mbox{\rm\textsf{\small After$\_$Execution}}}
\newcommand{\Comp}{\mbox{\rm\textsf{\small Compositionality}}}
\newcommand{\Synt}{\mbox{\rm\textsf{\small Syntactic$\_$Criteria}}}
\newcommand{\Possib}{\mbox{\rm\textsf{\small Possib}}}
\newcommand{\Probab}{\mbox{\rm\textsf{\small Prob}}}
\newcommand{\MyTac}{\mbox{\rm\textsf{\small MyTactics}}}
\newcommand{\Bisim}{\mbox{\rm\textsf{\small Bisim}}}
\newcommand{\Conc}{\mbox{\rm\textsf{\small Concrete}}}
\newcommand{\ind}{\mbox{$\;\sim\;$}}

\newcommand{\bis}{\mbox{$\;\approx\;$}}
\newcommand{\sbis}{\mbox{$\;\approx_{\tiny \textsf{S}}\,$}}
\newcommand{\zobis}{\mbox{$\;\approx_{\tiny \textsf{01}}\,$}}
\newcommand{\zobist}{\mbox{$\;\approx_{\tiny \textsf{01T}}\,$}}
\newcommand{\wbis}{\mbox{$\;\approx_{\tiny \textsf{W}}\,$}}
\newcommand{\wbist}{\mbox{$\;\approx_{\tiny \textsf{WT}}\,$}}
\newcommand{\bist}{\mbox{$\;\approx_{\tiny \textsf{T}}\,$}}
\newcommand{\LRA}{\Longrightarrow} 
\newcommand{\Lra}{\Longrightarrow}

\newcommand{\Atm}{\mbox{\rm\textsf{\small Atm}}}
\newcommand{\AAtm}{\mbox{\scriptsize \rm\textsf{Atm}}}
\newcommand{\Seq}{\mbox{\rm\textsf{\small Seq}}}
\newcommand{\SSeq}{\mbox{\scriptsize \rm\textsf{Seq}}}
\newcommand{\If}{\mbox{\rm\textsf{\small If}}}
\newcommand{\IIf}{\mbox{\scriptsize \rm\textsf{If}}}
\newcommand{\Ch}{\mbox{\rm\textsf{\small Ch}}}
\newcommand{\CCh}{\mbox{\scriptsize \rm\textsf{Ch}}}
\newcommand{\Par}{\mbox{\rm\textsf{\small Par}}}
\newcommand{\PPar}{\mbox{\scriptsize \rm\textsf{Par}}}
\newcommand{\ParT}{\mbox{\rm\textsf{\small Par\hspace{-0.1ex}T}}}
\newcommand{\PParT}{\mbox{\scriptsize \rm\textsf{ParT}}}
\newcommand{\While}{\mbox{\rm\textsf{\small While}}}
\newcommand{\WWhile}{\mbox{\scriptsize \rm\textsf{While}}}

\noindent
This is a formalization of the mathematical development presented in the paper \cite{pop-pos}: 
%
\begin{itemize}
\item a uniform framework where 
a wide range of language-based noninterference variants from the literature are expressed and 
compared w.r.t.~their {\em contracts}: 
the strength of the security properties they ensure 
weighed against  
the harshness of the syntactic conditions they enforce;  
%
\item syntactic criteria for proving that a program has a specific noninterference
property, using only compositionality, which captures uniformly several 
security type-system results from the literature and suggests a further improved type system.  
\end{itemize}
%
There are two auxiliary theories:
\begin{itemize}
\item \MyTac, introducing a few customized tactics; 
%
\item \Bisim, describing an abstract notion of bisimilarity relation, namely, the greatest 
symmetric relation that is a fixpoint of a monotonic operator--this shall be instantiated 
to several concrete bisimilarity later. 
\end{itemize}

  
\begin{figure}
$$
\xymatrix@C=0.5pc@R=1pc{
     \Synt   & \Conc & \Afte   \\
     \Comp \ar@{-}[u] \ar@{-}[ur] & &   \\
     & \Duri \ar@{-}[ul] \ar@{-}[uur]& \\
     &  \Lang \ar@{-}[u]&                 
}
$$
\vspace*{-3ex}
\caption{Main Theory Structure}
\label{fig-isabelle}
\vspace*{-3ex}
\end{figure} 

The main theories of the development (shown in Fig.~\ref{fig-isabelle}) are 
organized similarly to the sectionwise structure of \cite{pop-pos}:

\Lang\ corresponds to \S2 in \cite{pop-pos}.  It introduces and customizes the syntax and 
small-step operational semantics of a 
while language with parallel composition, using notations very similar to the paper.  

\Duri\footnote{``During-execution" (bisimilarity-based) noninterference should be contrasted with ``after-execution" (trace-based) 
noninterference according to the distinction made in \cite{pop-pos} at the begining of \S7.} 
mainly corresponds to \S3 in \cite{pop-pos}, defining the various coinductive notions 
from there: self isomorphism, discreetness, variations of strong, weak and 01-bisimilarity.  
Prop.~1 from the paper,  stating implications between these notions, 
is proved as the theorems bis$\_$imp and siso$\_$bis.\footnote{To help 
the reader distinguish the main results from the auxiliary lemmas, the former are marked 
in the scripts with the keyword ``theorem".} The bisimilarity inclusions stated in bis$\_$imp are slightly more general than those in Prop.~1, 
in that they employ the binary version of the relation,  e.g., $c \sbis d \LRA c \wbist d$ instead of $c \sbis c \LRA c \wbist c$.  

\Comp\ mainly corresponds to the homonymous \S4 in \cite{pop-pos}.  The paper's compositionality result, Prop.~2, is scattered through the theory 
as theorems with self-explanatory names, indication the compositionality relationship between notions of noninterference and language constructs, 
e.g., While$\_$WbisT (while versus termination-sensitive weak bisimilarity), Par$\_$ZObis (parallel composition versus $01$-bisimilarity).  

Theories \Duri\ and \Comp\ also include the novel notion of noninterference $\bist$ introduced in \S5 of \cite{pop-pos}, 
based on the ``must terminate" condition, which is given the same treatment as the other notions: 
bis$\_$imp in \Duri\ states the implication relationship between $\bist$ and the other bisimilarities (Prop.~3.(1) from \cite{pop-pos}), 
while various intuitively named theorems from \Lang\ state the compositionality properties of $\bist$ (Prop.~3.(2) from \cite{pop-pos}).  

\Synt\ corresponds to the homonymous \S6 in \cite{pop-pos}.  The syntactic analogues of the semantics notions, 
indicated in the paper by overlining, e.g., $\overline{\textsf{discr}}$, are in the scripts prefixed by ``SC" (from ``syntactic criterion"), e.g., SC$\_$discr, SC$\_$WbisT.  
Props.~4 and 5 from the paper (stating the relationship between the syntactic and the semantic notions 
and the implications between the syntactic notions, respectively) are again scattered through the theory under self-explanatory names.  

\Conc\ contains an instantiation of the 
indistinguishability relation $\!\!\ind\!\!$ from \cite{pop-pos} to the standard two-level security setting 
described in the paper's Example 2.  

Finally, \Afte\ corresponds to \S7 in \cite{pop-pos}, dealing with the after-execution guarantees of the during-execution 
notions of security. Prop.~6 in the paper is stated in the scripts as theorems Sbis$\_$trace, ZObisT$\_$trace and WbisT$\_$trace, 
Prop.~7 as theorems ZObis$\_$trace and Wbis$\_$trace, and  
Prop.~8 as theorem BisT$\_$trace.  
  

     


% generated text of all theories
\input{session}

\chapter{Conclusion}\label{ch:conclusion}

This work presented the Isabelle Collections Framework, an efficient and extensible collections framework for Isabelle/HOL.
The framework features data-refinement techniques to refine algorithms to use concrete collection datastructures,
and is compatible with the Isabelle/HOL code generator, such that efficient code can be generated for all supported target languages.
Finally, we defined a data refinement framework for the while-combinator, and used it to specify a state-space exploration algorithm
and stepwise refined the specification to an executable DFS-algorithm using a hashset to store the set of already known states.

Up to now, interfaces for sets and maps are specified and implemented using lists, red-black-trees, and hashing. Moreover, an amortized constant time 
fifo-queue (based on two stacks) has been implemented. However, the framwork is extensible, i.e. new interfaces, algorithms and implementations can easily be added and integrated with the existing ones.

\section{Future Work}
There are several starting points for future work:

\begin{itemize}
  \item Currently, the generic algorithms are instantiated semi-automatically by an ad-hoc ruby script and a
  manual description of the generic algorithms to be instantiated. However, the process of instantiating
  generic algorithms could be fully mechanized, if one would add support for specifying interfaces, 
  implementations and generic algorithms in Isabelle/HOL. 
  
  \item Generic algorithms are declared as functions that take the required functions as arguments.
  While this is convenient for small generic algorithms, it can become tedious for larger algorithms,
  involving many interfaces. 
  Luckily, the generic algorithms defined in this library are rather small. For bigger developments, we currently recommend to
  fix the used implementations, i.e. to define specific algorithms rather than generic ones. This is, for example, done in the 
  DFS-search example (Section~\ref{thy:Exploration_Example}), where we fix hashsets instead of defining a generic algorithm for
  all set implementations. Hence there is a need for exploring more convenient ways to specify generic algorithms.
  
%  \item What is called hashsets in this library are not true hashsets, as mapping hashcodes to buckets is done by a red-black tree instead
%      of an array. This is more convenient for proofs in Isabelle/HOL, and has the advantage that no rehashing is needed if the map grows bigger, but it is also slower than array-based hashmaps. There is support for true arrays in Isabelle by means of a Heap-Monad \cite{BKHEM08} with the support for efficient code generation, that could probably be used to implement hashmaps with arrays.
\end{itemize}

\section {Trusted Code Base}
  In this section we shortly characterize on what our formal proofs depend, i.e. how to interpret the information contained in this formal proof and the fact that it
  is accepted by the Isabelle/HOL system.

  First of all, you have to trust the theorem prover and its axiomatization of HOL, the ML-platform, the operating system software and the hardware it runs on.
  All these components are, in theory, able to cause false theorems to be proven. However, the probability of a false theorem to get proven due to a hardware error 
  or an error in the operating system software is reasonably low. There are errors in hardware and operating systems, but they will usually cause the system to crash 
  or exhibit other unexpected behaviour, instead of causing Isabelle to quitely accept a false theorem and behave normal otherwise. The theorem prover itself is a bit more critical in this aspect. However, Isabelle/HOL is implemented in LCF-style, i.e. all the proofs are eventually checked by a small kernel of trusted code, containing rather simple operations. HOL is the logic that is most frequently used with Isabelle, and it is unlikely that it's axiomatization in Isabelle is inconsistent and no one has found and reported this inconsistency yet.

  The next crucial point is the code generator of Isabelle. We derive executable code from our specifications. The code generator contains another (thin) layer of untrusted code. This layer has some known deficiencies\footnote{For example, the Haskell code generator may generate variables starting with upper-case letters, while the Haskell-specification requires variables to start with lowercase letters. Moreover, the ML code generator does not know the ML value restriction, and may generate code that violates this restriction.} (as of Isabelle2009) in the sense that invalid code is generated. This code is then rejected by the target language's compiler or interpreter, but does not silently compute the wrong thing. 

  Moreover, assuming correctness of the code generator, the generated code is only guaranteed to be {\em partially} correct\footnote{A simple example is the always-diverging function ${\sf f_{div}}::{\sf bool} = {\sf while}~(\lambda x.~{\sf True})~{\sf id}~{\sf True}$ that is definable in HOL. The lemma $\forall x.~ x = {\sf if}~{\sf f_{div}}~{\sf then}~x~{\sf else}~x$ is provable in Isabelle and rewriting based on it could, theoretically, be inserted before the code generation process, resulting in code that always diverges}, i.e. there are no formal termination guarantees.

  Furthermore, manual adaptations of the code generator setup are also part of the trusted code base.
  For array-based hash maps, the Isabelle Collections Framework provides an ML implementation for arrays with in-place updates that is unverified; for Haskell, we use the DiffArray implementation from the Haskell library.
  Other than this, the Isabelle Collections Framework does not add any adaptations other than those available in the Isabelle/HOL library, in particular Efficient\_Nat.

\section{Acknowledgement}
We thank Tobias Nipkow for encouraging us to make the collections framework an independent development. Moreover, we thank Markus M\"uller-Olm for discussion about data-refinement. Finally, we thank the people on the Isabelle mailing list for quick and useful response to any Isabelle-related questions.


\clearpage

% optional bibliography
\bibliographystyle{abbrv}
\bibliography{root}

\end{document}

%%% Local Variables:
%%% mode: latex
%%% TeX-master: t
%%% End:
