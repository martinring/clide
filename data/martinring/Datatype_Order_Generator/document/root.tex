\documentclass[11pt,a4paper]{article}
\usepackage{isabelle,isabellesym}

% further packages required for unusual symbols (see also isabellesym.sty)
% use only when needed
%\usepackage{amssymb}                  % for \<leadsto>, \<box>, \<diamond>,
                                       % \<sqsupset>, \<mho>, \<Join>, 
                                       % \<lhd>, \<lesssim>, \<greatersim>,
                                       % \<lessapprox>, \<greaterapprox>,
                                       % \<triangleq>, \<yen>, \<lozenge>
%\usepackage[greek,english]{babel}     % greek for \<euro>,
                                       % english for \<guillemotleft>, 
                                       %             \<guillemotright>
                                       % default language = last
%\usepackage[latin1]{inputenc}         % for \<onesuperior>, \<onequarter>,
                                       % \<twosuperior>, \<onehalf>,
                                       % \<threesuperior>, \<threequarters>,
                                       % \<degree>
%\usepackage[only,bigsqcap]{stmaryrd}  % for \<Sqinter>
%\usepackage{eufrak}                   % for \<AA> ... \<ZZ>, \<aa> ... \<zz>
                                       % (only needed if amssymb not used)
%\usepackage{textcomp}                 % for \<cent>, \<currency>

% this should be the last package used
\usepackage{pdfsetup}

% urls in roman style, theory text in math-similar italics
\urlstyle{rm}
\isabellestyle{it}

\newcommand\isafor{\textsf{IsaFoR}}
\newcommand\ceta{\textsf{Ce\kern-.18emT\kern-.18emA}}

\begin{document}

\title{Generating linear orders for datatypes.\footnote{Supported by FWF (Austrian Science Fund) project P22767-N13.}}
\author{Ren\'e Thiemann}
\maketitle

\begin{abstract}
  We provide a tactic which automatically synthesizes linear orders for datatypes
  as it is possible using Haskell's ``deriving Ord'' feature.
  The tactic is able to handle datatypes with mutual or higher-order recursion,
  if the nesting is not too complex. 

  Our method complements the Isabelle Collection Framework which for some 
  datastructures---like balanced trees---demands that the type of keys is 
  linearly ordered \cite{rbt}.
  
  Our formalization was performed as part of the \isafor/\ceta{} project%
  \footnote{\url{http://cl-informatik.uibk.ac.at/software/ceta}} \cite{CeTA}.
  With our new tactic we could completely remove 
  tedious proofs for linear orders of two datatypes.
\end{abstract}

\tableofcontents


% include generated text of all theories
\input{session}



\bibliographystyle{abbrv}
\bibliography{root}

\end{document}
