

\documentclass[11pt,a4paper]{article}
\usepackage{isabelle,isabellesym}

% further packages required for unusual symbols (see also isabellesym.sty)
% use only when needed
%\usepackage{amssymb}                  % for \<leadsto>, \<box>, \<diamond>,
                                       % \<sqsupset>, \<mho>, \<Join>, 
                                       % \<lhd>, \<lesssim>, \<greatersim>,
                                       % \<lessapprox>, \<greaterapprox>,
                                       % \<triangleq>, \<yen>, \<lozenge>
%\usepackage[greek,english]{babel}     % greek for \<euro>,
                                       % english for \<guillemotleft>, 
                                       %             \<guillemotright>
                                       % default language = last
%\usepackage[latin1]{inputenc}         % for \<onesuperior>, \<onequarter>,
                                       % \<twosuperior>, \<onehalf>,
                                       % \<threesuperior>, \<threequarters>,
                                       % \<degree>
%\usepackage[only,bigsqcap]{stmaryrd}  % for \<Sqinter>
%\usepackage{eufrak}                   % for \<AA> ... \<ZZ>, \<aa> ... \<zz>
                                       % (only needed if amssymb not used)
%\usepackage{textcomp}                 % for \<cent>, \<currency>

% this should be the last package used
\usepackage{pdfsetup}

% urls in roman style, theory text in math-similar italics
\urlstyle{rm}
\isabellestyle{it}


\begin{document}

\title{Data refinement of representation of a file}
\author{Karen Zee and Viktor Kuncak}
\maketitle

\begin{abstract}
  This document illustrates the verification of basic file
  operations (file creation, file read and file write) in
  Isabelle theorem prover \cite{LNCS2283}.  We describe a
  file at two levels of abstraction: an abstract file
  represented as a resizable array, and a concrete file
  represented using data blocks.
\end{abstract}

\tableofcontents

\parindent 0pt\parskip 0.5ex

\section{Introduction}

This document is based on
\cite{ArkoudasETAL04VerifyingFileSystemImplementationICFEM}, which
explores the challenges of verifying the core operations of a
Unix-like file system \cite{thompson78unix,mckusick84fast}.  The paper
\cite{ArkoudasETAL04VerifyingFileSystemImplementationICFEM} formalizes
the specification of the file system as a map from file names to
sequences of bytes, then formalizes an implementation that uses such
standard file system data structures as i-nodes and fixed-sized disk
blocks.  The correctness of the
implementation is verified by proving the existence of a simulation relation
\cite{RoeverEngelhardt98DataRefinement} between the specification and
the implementation.  The original effort of
\cite{ArkoudasETAL04VerifyingFileSystemImplementationICFEM} started in
Isabelle.  The process of developing the proof in Isabelle helped to 
remove the initial bugs in the concrete and
abstract models (though the proof has not been completed so far).  

Here we present a completed proof for a simplified problem:
data refinement of a single file.  We present operations on
both abstract and concrete files, define a function mapping
concrete files to abstract files, and prove that this
function is a simulation relation.

We use two libraries of arrays: arrays without bounds
checks, which can be thought of as an array with an
unbounded number of elements, and resizable arrays, which
have a notion of the current size, but expand in response to
array writes that are outside the current bounds.


% include generated text of all theories
\input{session}

\bibliographystyle{abbrv}
\bibliography{root}

\end{document}
