\documentclass[11pt,a4paper]{article}
\usepackage{graphicx}
\usepackage{isabelle,isabellesym,pdfsetup}
\usepackage{textcomp}

\urlstyle{rm}
\isabellestyle{it}

\begin{document}

\title{Some results of number theory}
\author{Jeremy Avigad\\
    David Gray\\
    Adam Kramer\\
    Thomas M Rasmussen}

\maketitle

\begin{abstract}
This is a collection of formalized proofs of many results of number theory.
The proofs of the Chinese Remainder Theorem and Wilson's Theorem are due to
Rasmussen.  The proof of Gauss's law of quadratic reciprocity is due to
Avigad, Gray and Kramer.  Proofs can be found in most introductory number
theory textbooks; Goldman's \emph{The Queen of Mathematics: a Historically
Motivated Guide to Number Theory} provides some historical context.

Avigad, Gray and Kramer have also provided library theories dealing with
finite sets and finite sums, divisibility and congruences, parity and
residues.  The authors are engaged in redesigning and polishing these theories
for more serious use.  For the latest information in this respect, please see
the web page \url{http://www.andrew.cmu.edu/~avigad/isabelle}.  Other theories
contain proofs of Euler's criteria, Gauss' lemma, and the law of quadratic
reciprocity.  The formalization follows Eisenstein's proof, which is the one
most commonly found in introductory textbooks; in particular, it follows the
presentation in Niven and Zuckerman, \emph{The Theory of Numbers}.

To avoid having to count roots of polynomials, however, we relied on a trick
previously used by David Russinoff in formalizing quadratic reciprocity for
the Boyer-Moore theorem prover; see Russinoff, David, ``A mechanical proof
of quadratic reciprocity,'' \emph{Journal of Automated Reasoning} 8:3-21,
1992.  We are grateful to Larry Paulson for calling our attention to this
reference.
\end{abstract}

\tableofcontents

\begin{center}
  \includegraphics[scale=0.5]{session_graph}  
\end{center}

\newpage

\parindent 0pt\parskip 0.5ex
\input{session}

\end{document}
