\chapter{Installationsanweisungen}
\label{sec:install}

Für die lokale Ausführung des Servers kann die auf dem beiliegenden USB-Stick bereitgestellte
Software verwendet werden. Dafür müssen zunächst folgende Voraussetzungen erfüllt sein:

\begin{enumerate}
  \item Die Isabelle 2012 Plattform muss installiert werden. Diese ist zusammen mit detaillierten 
  Installationsanweisungen zu finden unter:
  \begin{quote}
  http://isabelle.in.tum.de/
  \end{quote}
  Wichtig ist hierbei, dass die Umgebungsvariable \texttt{ISABELLE\_HOME} gesetzt ist, damit die 
  Anwendung den Ort der Distribution feststellen kann.
  \item Ein aktuelles \textit{Java Development Kit} (JDK) der Version 7 oder höher muss installiert und als 
  Standardplattform konfiguriert sein. (JDK < 7 führt zu unerwarteten Fehlermeldungen und Abstürzen, 
  da Inkompatibilitäten bestehen.)
  \begin{quote}
  http://www.oracle.com/technetwork/java/javase/downloads/index.html
  \end{quote}
  \item Das Play Framework in Version 2.1 (Nicht 2.0.1!!) ist zu installieren. Es muss 
  sichergestellt sein, dass \texttt{play} im Pfad steht und damit von der Konsole aus aufrufbar ist. 
  Zu diesem Zeitpunkt ist die aktuelle Version 2.1-RC1
  \begin{quote}
  http://download.playframework.org/releases/
  \end{quote}
\end{enumerate}

Um die Anwendung zu starten, muss zunächst in das Stammverzeichnis gewechselt werden (entweder der
USB-Stick oder eine Kopie davon auf der lokalen Platte).

Hier kann nun das Kommando

\begin{quote}
\texttt{> play start}
\end{quote}

ausgeführt werden, wodurch der Server gestartet wird. Mit \textit{CTRL-D} kann die Log-Ausgabe
verlassen werden.

Die Anwendung ist nun unter \textit{http://localhost:9000} vom Browser aus Verfügbar. Es wird
empfohlen, Google Chrome in Version 24 oder höher für die Betrachtung zu verwenden.

Standardmäßig ist der Benutzer \texttt{martinring} mit dem Passwort \texttt{secret} eingerichtet.
Weitere Nutzer können durch Anpassen der Datei \texttt{/data/.users} hinzugefügt werden.