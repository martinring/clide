% \newacronym{label}{kurz}{lang}
% Zugriff via \gls{label} und Co.
% 
% Beispiel:
% 
%   \newacronym{acr:da}{DA}{Diplomarbeit}
% 
% wird im Text
% 
%   Heute schreibe ich meine \gls{arc:da}. Diese \gls{arc:da}.
% 
% zu
% 
%   Heute schreibe ich meine Diplomarbeit (DA). Diese DA.
% 
% Nur bei der ersten Verwendung der Abkürzung wird diese in langer Form
% dargestellt. Weitere Vorkommen nutzen die Kurz-Version.
% 
% Neben \gls{} gibt es auch noch \Gls{} und \GLS{}, welche die Groß- und
% Klein-Schreibung beeinflussen. Beispiel:
% 
%   \newacronym{acr:http}{http}{Hypertext Transfer Protocol}
%   \gls{acr:http} --> http
%   \Gls{acr:http} --> Http
%   \GLS{acr:http} --> HTTP
% 
% Zur Vereinfachung gibt es auch die Befehle \acr{}, \Acr{} und \ACR{}. Diese
% Befehle sind äquivalent:
% 
%   \gls{acr:foo} <=> \acr{foo}
%   \Gls{acr:foo} <=> \Acr{foo}
%   \GLS{acr:foo} <=> \ACR{foo}
%

\newacronym{acr:jvm}{JVM}{Java Virtual Machine}
\newacronym{acr:html}{HTML}{Hypertext Markup Language}
\newacronym{acr:http}{HTTP}{Hypertext Transfer Protocol}
\newacronym{acr:ajax}{AJAX}{Asynchronous JavaScript and XML}
\newacronym{acr:dom}{DOM}{Document Object Model}
\newacronym{acr:css}{CSS}{Cascading Style Sheets}
\newacronym{acr:js}{JS}{JavaScript}
\newacronym{acr:sml}{SML}{Standard ML}
\newacronym{acr:json}{JSON}{JavaScript Object Notation}
\newacronym{acr:bson}{BSON}{Binary JSON}
\newacronym{acr:less}{LESS}{Less CSS}
\newacronym{acr:url}{URL}{Uniform Resource Loactor}
\newacronym{acr:ace}{ACE}{AJAX.org Cloud9 Editor}
\newacronym{acr:xml}{XML}{Extensible Markup Language}
\newacronym{acr:sbt}{sbt}{Simple Build Tool}
\newacronym{acr:isar}{Isar}{Intelligible semi-automated reasoning}
\newacronym{acr:ui}{UI}{User Interface}
\newacronym{acr:repl}{REPL}{read-eval-print-loop}