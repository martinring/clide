\chapter{Zusammenfassung und Ausblick}

Die aus dieser Arbeit resultierende Anwendung ist ein ganz neuer Ansatz und es handelt sich um
weitestgehend unbetretenes Terrain. Während der Zeit der Entwicklung wurden nach und nach immer mehr
webbasierte Entwicklungsumgebungen für Mainstream-Sprachen bekannt. Am meisten Aufsehen hat dabei
wohl \textit{koding.com} erregt, was im Moment in eine nicht öffentliche Beta-Phase eingetreten ist,
zu der für uns leider kein Zugang besteht. Diese neuen Web-IDEs sind zwar nicht ganz vergleichbar,
da für Standard Programmiersprachen, im Gegensatz zu Isabelle, beispielsweise das Syntax-
Highlighting vollständig im Browser stattfinden kann. Jedoch zeigt dieser Trend, dass sehr wohl ein
Bedarf an IDEs im Browser besteht. Weil diese Entwicklungen parallel stattfanden und keine
quelloffenen Vorbilder bekannt sind, musste das gesamte Konzept neu entwickelt werden. Da in der
momentan langsam endenden Findungsphase für den neuen HTML5 Standard eine sehr hohe Fluktuation im
Bereich der Webentwicklung besteht, fiel es schwer fundierte Designentscheidungen zu treffen. Neue
Webframeworks sprießen ständig aus dem Boden und bewährte Entwurfsmuster wie sie für klassische
Desktopanwendungen bestehen, sucht man vergebens. Alleine für das Problem der Modularisierung von
\acr{js}-Code gibt es unzählige Varianten, die sich alle als \glqq die eine Lösung\grqq\ verkaufen.
Dankbarer Weise wurdem durch die Gründung der Firma \textit{Typesafe} durch Martin Odersky selbst
einige Frameworks für Scala, darunter Play, zu Mitgliedern des \textit{Typesafe Stack} gekrönt und
damit deren Weiterentwicklung sichergestellt bzw. auch deren Dokumentation vorangetrieben. Dennoch
erschienen im Laufe der Entwicklung 2 neue Scala Versionen, welche die Binärkompatibilität gebrochen
haben, sowie eine neue Version von Play (2.1), die viele Probleme gelöst hat, aber auch dazu geführt
hat, dass viele halbfertige eigene Lösungen für die in Play 2.0 bestehende Probleme wieder
eingestampft werden mussten.

Trotzdem muss das Ergebnis einen Vergleich mit den bisherigen Lösungen nicht scheuen. Es gilt noch
einiges an der Benutzbarkeit und der Browserkompatibilität zu erweitern, die Infrastruktur dafür ist
aber nun vorhanden und für ein größeres Team sollte es kein Problem darstellen, diese Lösung zu
einem vollwertigen Ersatz für bisherige Werkzeuge auszubauen. Darüber hinaus gibt es einige
revolutionäre Neuerungen im Bereich der interaktiven Theorembeweiser, wie die Aufteilung in Server
und Client sowie die neuen Möglichkeiten der Visualisierung.

Für eine Produktive Arbeit mit den Daten wäre es sinnvoll, dass in der \textit{Proover Community}
verbreitete Versionsverwaltungssystem \textit{Mercurial} zu integrieren und damit den notwendigen
Austausch von Daten zu gewährleisten.

Die entwickelte IDE kann als Ansatz für die Implementierung einer universellen Entwicklungsumgebung
dienen. So ist es sehr gut denkbar den neuen \textit{Presentation Compiler} für Scala in die
Anwendung zu integrieren und damit auf einfache Weise eine Entwicklungsumgebung für Scala zu
erstellen. Vor allem aber wäre es interessant die für die im Bereich der Theorembeweise anderen
relevanten Sprachen wie SML, Lisp oder Haskell zu unterstützen und damit eine abgerundete
Arbeitsumgebung bereitzustellen.