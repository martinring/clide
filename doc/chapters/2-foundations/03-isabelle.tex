\section{Isabelle}

Isabelle ist ein generisches System welches vor allem zum interaktiven Beweisen von Theoremen unter
der Nutzung von Logik höherer Ordnung (HOL) eingesetzt wird. Isabelle ist in \acr{sml} implementiert
und stark davon abhängig. \cite{isabelle}

\subsection{Asynchrones Beweisen}

Seit Version 2009 von Isabelle wurde es möglich Beweis-Dokumente, bzw. Theorien nebenläufig zu
überprüfen. Das macht es realistisch Echtzeitinformationen, so wie sie eine IDE liefern sollte
verfügbar zu machen.

\subsection{Isabelle/Scala}

Seit 2010 existiert mit Isabelle/Scala eine neue Schnittstelle zu Isabelle welche auf der Sprache
Scala basiert. Isabelle/Scala stellt eine API zur Arbeit mit Isabelle bereit, welche die zur Nutzung
relevanten Teile der SML Implementierung in Scala abbilden. \cite{iscala}

Über statisch typisierte Methoden können die Dokumente modifiziert werden. Dafür wurde ein internes
XML-Basiertes Protokoll eingeführt, welches die Scala API mit der SML API verknüpft.
ÜDementsprechend sind auch die Informationen welche von Isabelle geliefert werden typisiert. Das
macht Isabelle/Scala in der Nutzung recht robust, da ein Großteil der Fehler bereits zur
Übersetzungszeit gefunden werden kann.

Isabelle/Scala wurde für und zusammen mit der Anwendung Isabelle/jEdit entwickelt. JEdit wurde unter
anderem deswegen gewählt, weil es eine sehr einfache API hat und somit das Projekt nicht zu sehr auf
den Editor konzentriert ist.
