\section{HTML5}

Für die Implementierung der Browseranwendung wird auf den aktuellen Entwurf des  zukünfitgen HTML5
Standards zurückgegriffen. \cite{html5}

\acr{html} ist eine Sprache die der strukturierten  Beschreibung von Webseiten dient. Die Sprache
wurde in ihrer ursprünglichen Form  von 1989-1992, lange vor dem sogenannten Web 2.0, von
Wissenschaftlern des  Europäischen Kernforschungsinstitut CERN entwickelt. Sie war der erste nicht
proprietäre globale Standard zur digitalen übertragung von Strukturierten  Dokumenten. Die Sprache
\acr{html} an sich ist nicht geeignet um dynamische Inhaltem  wie sie heute praktisch auf allen
modernen Webseiten vorkommen, zu beschreiben.

Der heutige HTML5-Standard geht weit über die Sprache \acr{html} selbst hinaus und  umfasst vor
allem auch die Scriptsprache \acr{js} und die darin verfügbaren  Bibliotheken, sowie das
sogenannte \acr{dom}, auf welches in  Scripten zugegriffen werden kann um den angezeigten Inhalt
dynamisch zu  verändern.

\subsection{Dokumenobjektmodell}

Das \acr{dom} dient bla bla...

\subsection{Cascading Stylesheets}

\acr{css} ist eine Sprache die der Definition von  Stilen bzw.
Stilvorlagen für die Anzeige von Webinhalten dient.

Durch die Trennung von \acr{html} und \acr{css} wird erreicht, dass \acr{html}-Dokumente sich auf  den
Inhalt einer Seite beschränken, während alle die grafische Anzeige  belangenden Aspekte in die
sogenannten Stylesheets in \acr{css} Dateien ausgelagert  werden.

\subsubsection{LESS}

\acr{css} hat einige Einschränkungen welche die Arbeit damit erschweren:

\begin{itemize} 
  \item Es ist nicht möglich Variablen zu definieren um Eigenschaften, welche an
vielen Stellen vorkommen nur einmal zu definieren. 
  \item Es fehlen Funktionsdefinitonen um ähnliche
oder abhängige Definitionen  zusammenzufassen und zu parametrisieren. 
  \item Die Hierarchie einer
\acr{css}-Datei ist flach obwohl die Definitionen geschachtelt sind. Das reduziert die lesbarkeit
der Dateien. 
  \item Wenn aus Gründen der Übersichtlichkeit \acr{css} in mehrere Dateien aufgeteilt
werden, müssen alle Dateien einzeln geladen werden, was zu längeren Ladezeiten führt. \end{itemize}

\textit{LESS} ist eine Erweiterung von \acr{css} welche unter anderem  Variablen- und
Funktionsdefinitionen, verschachtelte Definitionen sowie  Dateiimports erlaubt. Damit werden die
oben genannten Einschränkungen von \acr{css} aufgehoben.

\subsection{JavaScript}

\acr{js} ist eine dynamisch typisierte, klassenlose objektorientierte  Scriptsprache und

\subsubsection{CoffeeScript}

\textit{CoffeeScript} ist eine

\subsection{HTTP}

HTTP (Hypertext Transport Protocol)

\subsubsection{AJAX}

AJAX (Asynchronous JavaScript and XML) ist keine Bibliothek und auch kein  Standard sondern ein sehr
weit verbreitetes Konzept zur Übertragung von Daten  zwischen Browser und Webserver. Hierbei wird
das \acr{js}-Objekt \texttt{XMLHttpRequest} verwendet um während der Anzeige einer Webseite

\subsubsection{WebSockets}

Websockets sind ein in HTML5 neu eingeführter Standard zur bidirektionalen  Kommunikation zwischen
Browser und einem Webserver. Hierbei wird anders als bei  AJAX eine direkte TCP-Verbindung
hergestellt. Diese Verbindung kann sowohl von  Browser- als auch von Serverseite aus gleich
verwendet werden. Das macht es  unnötig, wie bei AJAX wiederholte Anfragen oder Anfragen ohne
Zeitbegrenzung zu  stellen um Informationen vom Server zu erhalten wenn diese Verfügbar werden. Ein
weiterer Vorteil gegenüber HTTP-Anfragen ist, dass durch die direkte permanente Verbindung kein
Nachrichtenkopf mehr nötig ist. Das macht es deutlich  effizienter viele kleine Nachrichten zu
versenden.

\subsection{JavaScript-Bibliotheken}

Über den HTML5 Standard hinaus werden für die strukturierung der Anwendung einige 
\acr{js}-Bibliotheken verwendet, welche im folgenden kurz erläutert werden.

\subsubsection{jQuery}

Die Bibliothek \textit{jQuery} ist ein defacto Standard in der Webentwicklung.  In erster Linie
erleichtert es den Zugriff auf das \textit{DOM}. 

\subsubsection{Backbone}

\textit{Backbone} ist eine Bibliothek, die der Strukturierung von \acr{js} Anwendungen dient.

\subsubsection{RequireJS}

Da JavaScript von Haus aus keine Möglichkeit der Modularisierung bietet, komplexe Anwendungen
jedoch ohne Modularisierung kaum wartbar bleiben, haben sich unterschiedliche Lösungsansätze für
dieses Problem entwickelt. Einer der  umfassendsten ist die Bibliothek \textit{RequireJS}.

Mit der Funktion \texttt{define} können Module in Form von Funktionsdefinitionen  definiert werden.
Alle lokalen Variablen in dem Modul sind anders als bei  normalen Scripten außerhalb nicht mehr
sichtbar, da sie innerhalb einer  Funktion definiert wurden. Das Funktionsergebnis ist das was nach
außen Sichtbar  ist. Das kann ein beliebiges Objekt (also auch eine Funktion) sein.

Die so definierten Module können Abhängigkeiten untereinander spezifizieren  indem der
\texttt{define}-Funktion eine Liste von Modulen übergeben wird, die  das aktuelle Modul benötigt.
Die \textit{RequireJS}-Bibliothek sorgt dann dafür,  dass diese Module geladen wurden bevor das
aktuelle Modul