\section{Scala}

Die Programmiersprache Scala ist eine an der École polytechnique fédérale de Lausanne (EPFL) von
einem Team um Martin Odersky entwickelte statisch typisiserte,  objektorientierte, funktionale
Sprache. In Scala entwickelte Programme laufen  sowohl in der \acr{jvm} als auch in der Common
Language Runtime (CLR). Die Implementierung für die CLR hinkt jedoch stark  hinterher und ist für
diese Arbeit auch nicht von Interesse. Die aktuelle Sprachversion ist Scala 2.10, welche auch im
Rahmen dieser Arbeit Anwendung findet.

Scala versucht von Anfang an den Spagat zwischen funktionaler und  objektorientierter
Programmierung. Hierbei ist es sowohl möglich  rein objektorientierten, als auch rein funktionalen
Code zu schreiben. Dadurch  entstehen für den Programmierer sehr große Freiheitsgrade und es ist
beispielsweise auch möglich imperativen und funktionalen Code zu mischen. Diese  Freiheit erfordert
eine gewisse Verantwortung seitens des Programmierers um  lesbaren und wartbaren Code zu erstellen.

Seit 2011 wird Scala von der durch Martin Odersky ins Leben gerufenen Firma
\textit{Typesafe}\footnote{http://www.typesafe.com} zusammen mit den Frameworks \textit{Akka}
(Abschnitt\,\ref{sec:akka}) und \textit{Play} (Abschnitt\,\ref{sec:play}) sowie des Buildtools
\textit{sbt} (Abschnitt\,\ref{sec:sbt}) kommerziell im sogenannten \textit{Typesafe Stack}
weiterentwickelt und unterstützt. Dadurch wird Scala zu einer ernstzunehmenden Sprache, die auch in
Zukunft noch schnell weiterentwickelt wird. Scala eignet sich nicht zuletzt durch die Aktoren-
Bibliothek Akka dafür hochskalierbare verteilte Systeme zu entwickeln. Firmen wie Twitter, LinkedIn,
Siemens, TomTom, Sony, Amazon und die NASA treiben die Entwicklung immer schneller voran und sorgen
dafür, dass eine große Infrastruktur um Scala herum entsteht. (Siehe auch\,\cite{scala})

\subsection{Sprachkonzepte}

Im Folgenden sollen die für diese Arbeit relevanten Konzepte der Sprache Scala kurz vorgestellt
werden. Dabei wird allerings auf Grundlagen der objektorierten und funktionalen Programmierung sowie
auf bereits aus Java bekannte Konzepte verzichtet.

\subsubsection{Traits}

Traits sind ein besonders wertvoller und wichtiger Bestandteil von Scala. Sie sind ein Mittelweg
zwischen einer abstrakten Klasse und einem Interface. Dabei ermöglichen sie wie Interfaces in Java
Mehrfachvererbung, können jedoch auch genau wie abstrakte Klassen schon implementierte Funktionen
enthalten. Zudem können Traits in beliebige Klassen mit dem Konstruktor eingemischt werden
(Sogenannte \textit{Mixins}).

So können Teile der Funktionalität, die immer wieder verwendet werden, aber nicht von konkreten
Klassen abhängig sind, getrennt implementiert werden. Das fördert einen aspektorientierten
Programmierstil und schafft zudem Übersichtlichkeit im Code.

\subsubsection{Implizite Parameter}

Implizite Parameter werden in Scala verwendet, um Parameter, die sich aus dem  Kontext eines
Funktionsaufrufs erschließen können, nicht explizit übergeben zu müssen. Eine Funktion \texttt{f}
besitzt hierbei zusätzlich zu den normalen  Parameterlisten auch eine implizite Parameterliste:

\begin{lstlisting}{caption={Implizite Parameter in Scala},label={lst:implicitp}}
f(a: Int)(implicit x: T1, y: T2)
\end{lstlisting}

In dem Beispiel hat die Funktion einen normalen Parameter \texttt{a} und zwei  implizite Parameter
\texttt{x} und \texttt{y}. Der Compiler sucht bei einem  Funktionsaufruf, der die beiden oder
einen der impliziten Parameter nicht  spezifiziert nach impliziten Definitionen vom Typ \texttt{T1}
bzw. \texttt{T2}.  Diese Definitionen werden im aktuellen Sichtbarkeitsbereich nach bestimmten
Prioritäten gesucht. Dabei wird zunächst im aktuellen Objekt, dann in den zu den Typen \texttt{T1} und
\texttt{T2} gehörenden Objekten und dann in den  importierten Namensräumen gesucht. Implizite
Definitionen haben die Form \texttt{implicit def/val/var x: T = ...} wobei der Name \texttt{x}
keine Rolle  spielt.

\subsubsection{Implizite Konvertierungen}

Des Weiteren existiert das Konzept der impliziten Konvertierungen (\textit{implicit conversions}).
Hierbei werden bei Typfehlern zur Kompilierzeit Funktionen mit dem Modifizierer \texttt{implicit}
gesucht, die den gefundenen Typen in den nötigen Typen umwandeln können.  Die Priorisierung
geschieht hierbei genauso wie bei impliziten Parametern. Ein Beispiel:

\begin{lstlisting}{caption={Implizite Konversionen in Scala},label={lst:implicitc}}
implicit def t1tot2(x: T1): T2 = ...
def f(x: T2) = ...
val x: T1 = ...
f(x)
\end{lstlisting}

Hier wird eine implizite Konvertierung von \texttt{T1} nach \texttt{T2}  definiert. Bei dem Aufruf
\texttt{f(x)} kommt es zu einem Typfehler, weil  \texttt{T2} erwartet und \texttt{T1} übergeben
wird. Dieser Typfehler wird  gelöst, indem vom Compiler die Konvertierung eingesetzt wird. Der Aufruf
wird also intern zu \texttt{f(t1tot2(x))} erweitert.

\subsubsection{Typklassen}

Mit Hilfe von impliziten Definitionen ist es möglich, die aus der Sprache \textit{Haskell} bekannten
Typklassen in Scala nachzubilden.

Typklassen erlauben es, Ad-hoc-Polymorphie zu implementieren. Damit ist es ähnlich wie bei
Schnittstellen möglich, Funktionen für eine Menge von Typen bereitzustellen. Diese müssen jedoch
nicht direkt von den Typen implementiert sein und können so auch nachträglich beispielsweise für
Typen aus fremden Bibliotheken definiert werden.

In Scala werden Typklassen als generische abstrakte Klassen oder Traits implementiert. Instanzen der
Typklassen sind implizite Objektdefinitionen, die für einen spezifischen Typen die Typklasse bzw.
die abstrakte Klasse implementieren. Eine Funktion für eine bestimmte Typklasse kann durch eine
generische Funktion  realisiert werden. Diese ist dann über einen oder mehrere Typen parametrisiert
und erwartet als implizites Argument eine Instanz der Typklasse für diese Typen,  also eine
implizite Objektdefinition. Wenn diese im Sichtbarkeitsbereich existiert, wird sie automatisch vom
Compiler eingesetzt.

Als Beispiel betrachten wir die Ordnung eine Typs. Zunächst definieren wir einen generischen Trait
\texttt{Ord}, der über eine \texttt{compare}-Funktion zum Vergleich zweier Werte verfügt.

\begin{lstlisting}
trait Ord[T] {
  def compare: (a: T, b: T): Int
}
\end{lstlisting}

Wollen wir nun für einen beliebigen bestehenden Typ eine Ordnung definieren, müssen wir lediglich
ein implizites Objekt bereitstellen, das \texttt{Ord} implementiert.

\begin{lstlisting}
implicit object FileOrd extends Ord[File] {
  def compare: (a: File, b: File): Int = ...
}
\end{lstlisting}

Eine generische Funktion, welche die Funktion \texttt{compare} aus der Typklasse \texttt{Ord}
verwendet, kann nun definiert werden, indem eine Instanz von \texttt{Ord} als impliziter Parameter
erwartet wird.

\begin{lstlisting}
def sort[T](elems: List[T])(implicit ord: Ord[T]): List[T] = ...
\end{lstlisting}

Die Funktion \texttt{sort} kann nun verwendet werden, um Dateien zu sortieren solange das implizite
Objekt \texttt{FileOrd} beim Aufruf sichtbar ist.

Das Konzept der Typklasen ist vor allem dort sehr hilfreich, wo es darum geht fremde  Bibliotheken um
eigene Funktionen zu erweitern.

\subsubsection{Dynamische Typisierung}
\label{sec:dyn}

Seit Scala 2.9 ist es möglich, Funktionsaufrufe bei Typfehlern dynamisch zur  Laufzeit auflösen zu
lassen. Damit die Typsicherheit nicht generell verloren  geht, ist es nötig den Trait
\texttt{Dynamic} zu implementieren um einen Typ als dynamisch zu markieren. Wenn die Typüberprüfung
dann bei einem Aufruf auf einem als dynamisch markierten Objekt fehlschlägt, wird der Aufruf auf
eine der Funktionen \texttt{applyDynamic}, \texttt{applyDynamicNamed}, \texttt{selectDynamic} und
\texttt{updateDynamic}  abgebildet. Diese Übersetzung geschieht nach folgendem Muster:

\begin{lstlisting}{caption={Abbildung von dynamischen Aufrufen in Scala},label={lst:dyn}}
x.method("arg")    =>  x.applyDynamic("method")("arg")
x.method(x = y)    =>  x.applyDynamicNamed("method")(("x", y))
x.method(x = 1, 2) =>  x.applyDynamicNamed("method")(("x", 1), ("", 2))
x.field            =>  x.selectDynamic("field")
x.variable = 10    =>  x.updateDynamic("variable")(10)
x.list(10) = 13    =>  x.selectDynamic("list").update(10, 13)
x.list(10)         =>  x.applyDynamic("list")(10)
\end{lstlisting}
 
Die dynamische Typisierung ist ein Sprachkonstrukt, das in Scala nur in Ausnahmefällen verwendet
werden sollte. Es erweist sich aber als sehr praktisch in der Interaktion mit dynamischen Sprachen
wie JavaScript sowie bei der Arbeit mit externen Daten, bei denen keine Typinformationen vorliegen.
(JSON, SQL, usw.)

\subsection{SBT}
\label{sec:sbt}

Das \acr{sbt}\footnote{http://www.scala-sbt.org/} wird seit 2011 von Typesafe weiterentwickelt und
ist das Standard-Werkzeug zur automatischen Projekt- und Abhängigkeitsverwaltung in der Scala
Progammierung. Da es selbst in Scala geschrieben wurde und auch die Konfiguration in Scala-Objekten
stattfindet, ist es leicht, Erweiterungen dafür zu entwickeln. \Acr{sbt} ist mit Maven, einem sehr
verbreiteten Build Tool für Java kompatibel, sodass auch Javabibliotheken, welche in einem Maven-
Repository liegen als Abhängigkeiten definiert werden können. Typesafe stellt zudem eine sehr große
Auswahl von Scala-Bibliotheken in einem eigenen Repository zur Verfügung. Es können aber auch
beispielsweise öffentliche Git-Repositories als Abhängigkeit definiert werden.

\subsection{Akka}
\label{sec:akka}

Akka\footnote{http://www.akka.io/} war ursprünglich eine Implementierung des aus Erlang bekannten
Aktoren Modells, ist mitlerweile jedoch zu einem umfangreichen Framework zur Entwicklung von
hochperformanten verteilten Systemen gewachsen. Die Grundlage bilden nach wie vor Aktoren, wenn auch
in, gegenüber anderen Implementierungen, leicht veränderter Form. \cite{actors}

\subsubsection{Aktoren}

Aktoren haben sich als eine sehr wertvolle Abstraktion zur Modellierung von nebenläufigen Systemen
herausgestellt. Dabei wird die Software in mehrere, parallel agierende Aktoren aufgeteilt, die
sich untereinander Nachrichten senden. Um ungewollte Nebenläufigkeitseffekte auszuschließen, müssen
die Nachrichten unveränderbar sein, was in Scala bislang leider noch nicht überprüfbar ist und somit
in der Verantwortung des Entwicklers liegt.

In Akka sind Aktoren ortsunabhängig und können an einer beliebigen Stelle ausgeführt werden. Ein
Aktor kann auf dem selben Prozessor, einem anderen Prozessor im selben Rechner, auf einem anderen
Rechner im lokalen Netz oder auch auf einem beliebigen über das Internet erreichbaren Rechner
irgendwo auf der Welt ausgeführt werden. In dieser Eigenschaft liegt der Schlüssel zur
Skalierbarkeit: In Akka entwickelte Systeme können ohne Veränderungen auf einem oder tausenden
Rechnern gleichzeitig ausgeführt werden, ganz im Sinne des \textit{Cloud Computing}.

\subsubsection{Iteratees}
\label{sec:iteratees}

Akka bietet über die Aktoren und deren Verwaltung hinaus noch viele weitere hilfreiche
Abstraktionen. Besonders erwähnenswert sind hier noch die sogenannten \textit{Iteratees}. Iteratees
sind eine Möglichkeit einen Datenstrom zu verarbeiten, ohne dass alle Daten verfügbar sind. Das ist
dort besonders wichtig, wo es um nicht-blockierende kommunizierende Prozesse geht, wie es bei
Webanwendungen üblich ist. Eine gute Einführung in Iteratees ist in \cite{iteratees} zu finden.

\subsubsection{Futures}
\label{sec:futures}

Ein \textit{Future} ist ein Proxy für das Ergebnis einer Berechnung welches nicht unmittelbar
bekannt ist. Futures finden dann Anwendung, wenn Rechnungen nicht blockierend Ausgeführt werden
sollen und die Ergebnisse erst zu einem Späteren Zeitpunkt benötigt werden. Die Rechnung wird dabei
nebenläufig Ausgeführt. Da Futures in Scala 2.10 einen Monaden bilden, können wie komponiert werden.

\begin{lstlisting}
val a = future[Int](expensiveComputation1)
val b = future[Int](expensiveComputation2)
val c = for { aResult <- a, bResult <- b } yield aResult + bResult
\end{lstlisting}

Bislang existierten Futures in zwei Ausprägungen. Zum Einen gab es eine Implementierung in der Scala
Standardbibliothek zum anderen eine in Akka. Der Grund dafür war, dass die Futures in Scala einige
Schwächen hatten, welche erst nach Veröffentlichung erkannt wurden. So war es nicht leicht möglich
Futures zu kombinieren. In Scala 2.10 wurde eine überarbeitete Implementierung der Futures
eingeführt und das Akka Team hat sich daraufhin entschieden in der aktuellen Version 2.1 des Akka
Frameworks auf die Futures der Standardbibliothek zurückzugreifen.

\subsection{Play Framework}

Das \textit{Play Framework}\footnote{http://www.playframework.org} ist ein Rahmenwerk zur
Entwicklung von Webanwendungen auf der JVM in Java mit einer speziellen API für Scala. Play ist ein
sehr effizientes zustandsfreies Framework welches auf Akka aufbaut um hohe Skalierbarkeit zu
gewährleisten. Damit wird es leichter verteilte hochperformante Webanwendungen zu realisieren.

Die Struktur einer Play Anwendung ähnelt der bewährten Struktur von \textit{Ruby on
Rails}\footnote{http://rubyonrails.org/}. Es existieren \textit{Modelle}, \textit{Views} und
\textit{Controller}. Auch der Workflow ist ein ähnlicher. So ist es möglich während der
Entwicklung der Anwendung durch auffrischen der Seite im Browser immer die neuste Version zu sehen.
Play nutzt dafür \acr{sbt} als Build System.

Views werden als spezielle HTML-Templates, die auch Scala Code zulassen, definiert. Durch die
Ausdrucksorientiertheit von Scala ist es damit möglich, die Views typsicher zu halten und somit schon
zur Kompilierzeit über sehr viele Klassen von Fehlern informiert zu werden. Das ist ein klarer
Vorteil gegenüber den meisten andern modernen Webframeworks.

Modelle können natürlicherweise als beliebige Scala-Klassen bzw. Datenstrukturen repräsentiert
werden. Play beschränkt den Entwickler auch nicht auf eine bestimmte Möglichkeit der
Datenpersistenz.

Die Controller gruppieren Aktionen, die als Antworten auf Anfragen agieren und sind prinzipiell
als Unterklassen von \texttt{play.mvc.controller}, welche in der Akka Infrastruktur leben,
realisiert. Aktionen sind nicht mehr als Scala Funktionen, die die Parameter einer Anfrage
verarbeiten und daraus eine Antwort produzieren, welche an den Browser zurückgesendet wird.

Das Routing geschieht über die Konfigurationsdatei \texttt{conf/routes}, in welcher von URLs auf
Aktionen abgebildet wird. Es ist möglich, sogenanntes \textit{reverse routing} zu betreiben um von
einer Scala-Funktion zu einer URL zu gelangen. Des weiteren besteht die Möglichkeit, JavaScript-Code 
zu generieren, der im Browser zum reverse routing verwendet werden kann.

(Siehe auch Abschnitte\,\ref{sec:less},\,\ref{sec:coffeescript} sowie\,\ref{sec:requirejs})

\label{sec:play}