\section{Isabelle}

Isabelle ist ein generisches System welches vor allem zum interaktiven Beweisen von Theoremen unter
der Nutzung von Logiken höherer Ordnung eingesetzt wird. Isabelle ist in \acr{sml} implementiert und
stark davon abhängig. In Beweisen kann die volle Mächtigkeit von \acr{sml} an jeder Stelle benutzt
werden. Dadurch ist es schwer eine Echtzeitverarbeitung wie sie für eine Entwicklungsumgebung nötig
wäre zu realisieren. \cite{isabelle}

Die \acr{isar}-Platform bietet eine zusätzliche Abstraktion vom nackten \acr{sml} Code die dem
Benutzer eine komfortablere Umgebung zum formulieren von \textit{Beweisdokumenten} liefert. Darüber
hinaus ermöglichen die  strukturierten Dokumente eine Menschenlesbare Veröffentlichung der Beweise.
Das ist ein klarer Vorteil gegenüber Beweisen in \acr{sml} Skripten, welche eher Maschinenbezogen
sind.

Isabelle/Isar erlaubt das Veröffentlichen in verschiedene Formate, wie HTML und LaTeX. Dabei werden
Bestimmte Konstrukte besonders Dargestellt. Solche Symbole werden als
\texttt{\textbackslash\textless ...\textgreater} im Code repräsentiert. Es gibt theoretisch
unendlich viele dieser Symbole, allerdings wird nur eine kleine Menge von Symbolen in
\cite{isabelle} genau spezifiziert. Des weiteren existieren Kontrollzeichen in der Form
\texttt{\textbackslash\textless\textasciicircum ...\textgreater} welche genutzt werden können um
Sub- und Superskript zu repräsentieren bzw. Zeichen Fett darzustellen.

\subsection{Asynchrones Beweisen}

Seit Version 2009 von Isabelle wurde es möglich Beweisdokumente, bzw. Theorien nebenläufig zu
überprüfen. Das macht es realistisch Echtzeitinformationen während der Bearbeitung verfügbar zu
machen. \cite{parproof} 

\subsection{Isabelle/Scala}

Seit 2010 existiert mit Isabelle/Scala eine neue Schnittstelle zur Isabelle-Platform welche auf der
Sprache Scala basiert. Isabelle/Scala stellt eine API zur Arbeit mit Isabelle bereit, welche die zur
Nutzung relevanten Teile der SML Implementierung in Scala abbilden. \cite{iscala}

Über statisch typisierte Methoden können die Dokumente modifiziert werden. Dafür wurde ein internes
XML-Basiertes Protokoll eingeführt, welches die Scala API mit der SML API verknüpft.
ÜDementsprechend sind auch die Informationen welche von Isabelle geliefert werden typisiert. Das
macht Isabelle/Scala in der Nutzung recht robust, da ein Großteil der Fehler bereits zur
Übersetzungszeit gefunden werden kann.

Isabelle/Scala wurde für und zusammen mit der Anwendung Isabelle/jEdit entwickelt. JEdit wurde unter
anderem deswegen gewählt, weil es eine sehr einfache API hat und somit das Projekt nicht zu sehr auf
den Editor konzentriert ist.

\TODO{Isabelle}