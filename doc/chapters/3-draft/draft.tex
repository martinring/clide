\chapter{Entwurf}

Auf Grund der Komplexität und dem unvermeidbaren Rechenaufwand, welcher ein Theorembeweiser
mitbringt, ist es aus aktueller Sicht nahezu ausgeschlossen diese Arbeit zu größeren Teilen im
Browser zu verwirklichen. Die einzige von allen größeren Browsern unterstützte Scriptsprache ist zur
Zeit immernoch \textit{JavaScript}, welche vor allem auf Grund der dynamischen Typisierung und der
fehlenden Paralellisierbarkeit um einige Faktoren langsamer ausgeführt wird, als nativer Code.

Ein besonderer Vorteil, den die Anwendung gegenüber bisherigen Lösungen bringen soll, ist die
Mobilität. Das bedeutet, dass es von jedem Rechner mit Internetzugang und einem modernen Webbrowser
aus möglich sein soll, die Anwendung zu nutzen und auf eventuell bereits zu einem früheren Zeitpunk
an einem anderen Ort erstellten Theorien zugreifen zu können. Damit wird klar, dass die Projekte und
Theorien nicht lokal auf den einzelnen Rechnern verwaltet werden können, sondern an einer zentralen
von überall erreichbaren Stelle gespeichert sein müssen. Die Entscheidung zu einem Client-Server
Modell ergibt sich bei einer Webanwendung ohnehin automatisch.

Da es sich bei der Webanwendung um eine Entwicklungsumgebung handelt, welche insbesondere durch
Echtzeitinformationen einen Mehrwert bringen soll, ist einer der wichtigsten Aspekte des Entwurfs
die effiziente Kommunikation zwischen Server und Browser. Da die Kommunikation bei einer
Webanwendung generell sehr Zeitaufwändig ist - abhängig von der Internetanbindung kann es zu großen
Verzögerungen kommen - muss abgewogen werden, welche Arbeit im Browser und welche auf dem Server
erledigt werden soll. Als illustratives Beispiel kann das Syntax-Highlighting genannt werden:
Isabelle verfügt über eine innere und eine äußere Syntax, die sich im analytischen Aufwand stark
unterscheiden. Während die äußere Syntax relativ leicht zu parsieren ist und dabei bereits viele
Informationen liefert, ist die innere Syntax sehr komplex, flexibel und stark vom jeweiligen Kontext
abhängig. Somit liegt es nahe, das Syntaxhighlighting aufzuteilen: Um Übertragungskapazität zu
sparen kann das Highlighting der äußeren Syntax bereits im Browser stattfinden. Die feiner
granulierten Informationen aus der inneren Syntax können dann auf dem Server ermittelt werden und
mit kurzer Verzögerung in die Darstellung integriert werden.

\section{Server}

Der Webserver muss neben den normalen Aufgaben eines Webservers, wie der Bereitstellung der Inhalte,
der Authentifizierung der Benutzer sowie der Persitierung bzw. Bereitstellung der Nutzerspezifischen
Daten (In unserem Fall Sitzungen / Theorien), auch eine besondere Schnittstelle für die Arbeit mit
den Theorien bereitstellen. Vom Browser aus muss es möglich sein, die einzelnen Theorien in echtzeit
zu bearbeiten sowie Informationen über Beweiszustände bzw. Fehler als auch über die Typen, bzw.
Definitionen von Ausdrücken zu erhalten.

\subsection{Wahl des Webframeworks}

Für die Realisierung des Webservers wählen wir das \textit{Play
Framework}\footnote{http://www.playframework.org} in Version 2.1. Da wir die Isabelle/Scala
Schnittstelle nutzen, liegt es nahe ein Webframework in Scala zu nutzen um den Aufwand für die
Integration gering zu halten. Als Alternative existiert das \textit{Lift
Webframework}\footnote{http://www.liftweb.org} welches allerdings auf Grund des Rückzugs von David
Pollack aus der Entwicklung seit einiger Zeit nicht mehr geordnet weiter entwickelt wird und zudem
für unsere Zwecke überdimensioniert ist. Da die Webanwendung eher unkonventionelle Anforderungen an
den Server hat, nutzen die meisten Funktionen von \textit{Lift} uns nicht. \textit{Play} ist
hingegen vorallem auf hohe Performance und weniger auf die Lösung möglichst vieler Anwendungsfälle
ausgelegt, womit es für unsere Zwecke interessanter bleibt.

\subsection{Authentifizierung}

Die Authentifizierung soll in diesem Projekt bewusst einfach gehalten werden, da es sich hierbei um
eine Nebensächlichkeit handelt, welche ohne weitere Probleme aufgerüstet werden kann. Wir
beschränken uns daher auf die Möglichkeit sich mit einem Benutzernamen sowie einem dazugehörigen
Passwort einzuloggen, welche dann mit einer Konfigurationsdatei auf dem Server abgeglichen wird. Wir
können dann auf die Möglichkeit des Play Frameworks

\subsection{Persistenz}

Auch bei der Datenpersistenz 

\subsection{Isabelle/Scala integration}


\section{Kommunikation}


\section{Client}

\subsection{Benutzeroberfläche}

\subsection{Client-Modell}

\subsection{}