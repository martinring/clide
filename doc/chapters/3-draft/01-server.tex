\section{Server}

Der Webserver muss neben den normalen Aufgaben eines Webservers, wie der Bereitstellung der Inhalte,
der Authentifizierung der Benutzer sowie der Persitierung bzw. Bereitstellung der Nutzerspezifischen
Daten (In unserem Fall Sitzungen / Theorien), auch eine besondere Schnittstelle für die Arbeit mit
den Theorien bereitstellen. Vom Browser aus muss es möglich sein, 

\begin{itemize}
  \item die einzelnen Theorien in Echtzeit zu bearbeiten,
  \item Informationen über Beweiszustände bzw. Fehler zu erhalten,
  \item als auch Informationen über die Typen, bzw. Definitionen von Ausdrücken zu erhalten.
\end{itemize}

All diese Informationen müssen zuvor Serverseitig aufbereitet und bereitgestellt werden. Dabei ist
es aus Sicht der Perfomanz wichtig, unnötige Informationen zu eliminieren und die Daten zu
komprimieren.

\TODO{Weitere Anforderungen an den Server}

\subsection{Wahl des Webframeworks}

Für die Realisierung des Webservers wählen wir das \textit{Play
Framework}\footnote{http://www.playframework.org} in Version 2.1. Da wir die Isabelle/Scala
Schnittstelle nutzen, liegt es nahe ein Webframework in Scala zu nutzen um den Aufwand für die
Integration gering zu halten. Als Alternative existiert das \textit{Lift
Webframework}\footnote{http://www.liftweb.org} welches allerdings auf Grund des Rückzugs von David
Pollack aus der Entwicklung seit einiger Zeit nicht mehr geordnet weiter entwickelt wird und zudem
für unsere Zwecke überdimensioniert ist. Da die Webanwendung eher unkonventionelle Anforderungen an
den Server hat, nutzen die meisten Funktionen von \textit{Lift} nichts. Das \textit{Play}
Webframework ist hingegen vorallem auf hohe Performance und weniger auf die Lösung möglichst vieler
Anwendungsfälle ausgelegt, womit es für unsere Zwecke interessanter bleibt.

\subsection{Authentifizierung}

Die Authentifizierung soll in diesem Projekt bewusst einfach gehalten werden, da es sich hierbei um
eine Nebensächlichkeit handelt, welche ohne weitere Probleme aufgerüstet werden kann. Wir
beschränken uns daher auf die Möglichkeit sich mit einem Benutzernamen sowie einem dazugehörigen
Passwort einzuloggen, welche dann mit einer Konfigurationsdatei auf dem Server abgeglichen wird. Wir
können dann auf die Möglichkeit des Play Frameworks zur sicheren Verwaltung von Session-Bezogenen
Daten zurückgreifen um die Anmeldung aufrechtzuerhalten.

\subsection{Persistenz}

Als Besonderheit bei der Datenpersistenz sind die serverseitig zu verwaltenden Theorien zu nennen.
Da jeder Benutzer eine von allen anderen Benutzern unabhängige Menge von Projekten mit Theorien
besitzt, also eine hierarchische Struktur besteht, spricht nichts dagegen, die Daten Serverseitig im
Dateisystem zu verwalten. Somit ist auch eine eventuelle spätere Integration eines
Versionsverwaltungssystems möglich. Da über diese Daten hinaus nur wenige Informationen vom Server
verwaltet werden müssen, ist die Einrichtung und Anbindung einer Datenbank nicht von Nöten.

\subsection{Bereitstellung von Resourcen}

Das Play-Framework bietet ausgefeilte Möglichkeiten sowohl statische als aus dynamische Resourcen
bereit zu stellen. Ein Hauptaugenmerk liegt hierbei auf der Bereitstellung der nötigen \acr{js}-,
\acr{css}- sowie \acr{html}-Dateien.

Während der Entwicklung dieser Arbeit wurde ein Modul für das Play Framework entwickelt, um \textbf
{CoffeeScript-Dateien} automatisch zu \acr{js} zu übersetzen, Abhängigkeiten mit RequireJS
aufzulösen und eine Optimierte \acr{js}-Datei bereitzustellen. Da in der in Kürze erscheinenden
Version 2.1 des Play Framework ganau diese Funktionalität entwickelt wurde, liegt die Entscheidung
nahe, diese Funktionalität zu nutzen und das eigene Modul wegfallen zu lassen. (Siehe
\ref{sec:play})

Ebenfalls von Play unterstützt wird die Möglichkeit  \textbf{\acr{less}-Dateien} mit ihren
Abhängigkeiten zu einer \acr{css}-Datei zu übersetzen. Da diese genau wie bei der CoffeeScript-
Übersetzung zur Entwicklungszeit in lesbare und im Produktiveinsatz in optimmierte Dateien übersetzt
werden, ist dies eine optimale Wahl.

\textbf{Statische Resourcen} wie in unserem Fall z.B. Font-Dateien oder fremde \acr{js}-Bibliotheken
werden durch einen sogenannten Asset-Controler aus dem Play-Framework bereitgestellt. Dieser bietet
die Möglichkeit alle Dateien in einem Ordner statisch bereitzustellen. In unserem Fall sind das die
Dateien im Ordner \texttt{"/public"} welche unter der \acr{url} \texttt{"/assets"} bereitgestellt
werden.

\subsection{Isabelle/Scala Integration}

\TODO{Isabelle/Scala auf dem Server}
