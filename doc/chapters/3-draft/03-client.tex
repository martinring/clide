\section{Client}

Zur Realisierung des Clients wird auf die etwas komforablere Scriptsprache CoffeeScript
zurückgegriffen. CoffeeScript-Code is gegenüber \acr{js}-Code deutlich kürzer, (Etwa 30\%) und hat
eine an Funktionale Sprachen wie Haskell erinnernde Syntax. Da mit den neuen Möglichkeiten in Play
2.1 CoffeeScript problemlos verwendet werden kann (Der Browser erhält kompiliertes \acr{js}) ,
entstehen hierdurch keine Nachteile.

Zur Strukturierung wird RequireJS sowie BackboneJS (und damit implizit auch UnderscoreJS) verwendet.
Durch die vielfältigen Möglichkeiten dieser Bibliotheken ist es möglich den Code klar zu
modularisieren.

\subsection{Browserkompatibilität}
\label{sec:comp}

Auf Grund der in \ref{sec:ws} beschriebenen Notwendigkeit, kann auf WebSockets nicht verzichtet
werden. Damit sind die meisten älteren Browser nicht mit der Anwendung kompatibel. 

\begin{table}[h]
\centering
\caption{Kompatibilität der gängigsten Browser mit den Verwendeten Standards}
\begin{tabular}{rlllll}
                  & \textbf{Chrome} & \textbf{Safari} & \textbf{IE} & \textbf{Firefox} 
                  & \textbf{Opera} \\\hline
  WebSockets      & 14.0            & 6.0             & 10.0        & 11.0             & 12.1  \\
  History API     & 5.0             & 5.0             & 10.0        & 4.0              & 11.5  \\
  WebSockets      & 4.0             & 4.0             & 10.0        & 3.5              & 10.6  \\
  CSS Transitions & 4.0             & 3.1             & 10.0        & 4.0              & 10.5  \\
\end{tabular}
\label{tab:comp}
\end{table}

\TODO{Kompatibilitäts-Tabelle}

Aus Tabelle \ref{tab:comp} ist zu entnehmen, dass alle weiteren in der Anwendung benutzten Standards
eine geringere oder die gleicher Anforderung an die Aktualität des Browsers haben. Da WebSockets ein
sehr neues Konzept sind,  dienen sie als Orientierung: Alle Features, welche von jedem Browser, der
WebSockets unterstützt, auch unterstützt werden, dürfen verwendet werden. Alle anderen schließen wir
aus, da sonst die Zahl der potentiellen Nutzer weiter eingeschränkt würde. Die Anwendung ist damit
im Standardbrowser auf allen Systemen mit einem aktuellen Betriebssystem (Windows 8, Ubuntu 12.10,
OpenSUSE 12.2, OS X 10.8.2), sowie auf dem iPad, und aktuellen Windows RT Tablets benutzbar. Bei der
Entwicklung wurde jedoch besonderes Augenmerk auf WebKit-basierte Browser, insbesondere Google
Chrome gelegt und einige der anderen genannten Systeme sind ungetestet und damit ohne Gewähr.

\subsection{Benutzeroberfläche}



\subsubsection{Die Editor-Komponente}

Die wichtigste Benutzerkomponente einer Entwicklungsumgebung ist der Text-Editor. Ein Editor für
Isabelle-Code hat hierbei besondere Anforderungen: Während in der Praxis bislang nur rudimentäre
Unterstützung für die Darstellung von Isabelle-Sonderzeichen und insbesondere von Sub- und
Superskript existierte, hat Isabelle/jEdit bereits eine stärkere Integration dieser eigentlich recht
essentiellen Visualisierungen eingeführt. Da bei der \acr{html}-Darstellung kaum Grenzen gesetzt
sind und sich \acr{css}-Formatierung sehr leicht dazu benutzen lässt bestimmte Text-Inhalte
besonders darzustellen, ist es klar, dass unsere Entwicklungsumgebung an dieser Stelle besonders
glänzen soll.

In einem ersten Prototypen war es möglich eine \acr{js}-Komponente zu entwickeln, welche es zuließ,
Isabelle-Code zu bearbeiten, sodass Sub- und Superskript sowie die Sonderzeichen korrekt dargestellt
wurden und bearbeitet werden konnten. Die besondere Anforderung bei ist hierbei nicht die
Darstellung sondern vor allem der Umgang mit den variablen Breiten. Selbst wenn ein Monospace-Font
verwendet würde, besteht das Problem, dass z.b. bei Sub- und Superskript nach Typographischen
Standards nur 66\% der Textgröße verwendet wird und somit auch die Zeichenbreite geringer wird. Da
aber eben die Visualisierung eine besondere Stärke der Anwendung sein soll, wollen wir zusätzlich
auch nicht darauf verzichten Ähnliche Fonts zu verwenden, wie in der Ausgabe der LaTeX-Dateien, also
auch Mathematische Sonderzeichen nicht in ein Raster quetschen. 

Eine weitere besondere Anforderung, welche bislang relativ einmalig zu sein
scheint, ist die Tatsache, dass das Syntax-Highlighting zu Teilen auf dem Server stattfindet und
somit eine Möglichkeit bestehen muss diese zusätzlichen Informationen in die Darstellung zu
integrieren.

Zusammenfassend können folgende besondere Anforderungen an die Editor-Komponente formuliert werden:
Der Editor muss in der Lage sein

\begin{itemize}
  \item Syntaxhighlighting zu betreiben,
  \item Externes Syntaxhighlighting verzögert zu integrieren,
  \item Schriftarten mit variabler Zeichenbreite anzuzeigen,
  \item Tooltips für Typinformationen o.ä. anzuzeigen und
  \item Isabelle-Sonderzeichen zu substituieren.
\end{itemize}

Da der Hauptsächliche Aufwand bei einer Editor-Komponente nicht darin liegt Text zu bearbeiten und
darzustellen sondern vor allem in der Infrastruktur drumherum (Copy/Paste, Suche, Selektieren,
Drag'n'Drop, usw.) ist es verlockend, eine fertige Komponente zu verwenden. Hier existieren mehrere
ausgereifte Alternativen. Bei genauerer Betrachtung gibt es jedoch keine, welche optimal für unsere
Zwecke geeignet ist. 

\begin{description} 

\item {Der \textbf{MDK-Editor}\footnote{http://www.mdk-photo.com/Editor/}} bietet viele Features, wird
aber seit 2008 nicht mehr weiter entwickelt und scheidet damit sofort aus.

\item {Der \textbf{\acr{ace}}\footnote{http://ace.ajax.org/}} (Ehemals Mozilla SkyWriter) wird
momentan sehr stark weiter entwickelt. \acr{ace} bietet bereits ein ausgeklügeltes Framework für das
Syntaxhighlighting, welches sich in einem Prototyp relativ leicht an das Serverseitige
Syntaxhighlighting anbinden ließ. \acr{ace} bietet alle Funktionen, welche man von einem Modernen
Text-Editor erwartet, hat jedoch einen entscheidenden Nachteil: Zur Darstellung wird aus
Performance-Gründen intern ein festes Raster verwendet. Dabei wird davon ausgegangen, dass ein
Monospace Font verwendet wird. Von diesem wird einmalig eine Zeichenbreite ermittelt und diese feste
Metrik wird dann für alle internen Operationen verwendet. Da diese Designentscheidung so
tiefgreifend ist, scheint es nicht realistisch in akzeptabler Zeit, die Komponente so zu
modifizieren, dass variable Breiten unterstützt werden können. Außerdem ist es nicht möglich
Textstellen durch Sonderzeichen zu Substituieren. Somit scheidet auch \acr{ace} für die Verwendung
in der Anwendung aus.

\item {\textbf{CodeMirror}\footnote{http://codemirror.net/}} ist ebenfalls eine weit entwickelte
Editor-Komponente, welche nicht ganz so umfangreich, wie acr{ace} ist, jedoch um einiges flexibler
erscheint. In einem Prototyp möglich einige eigene Modifikationen für die Darstellung (Sub-
Superskript, Tooltips, Hyperlinks) zu integrieren. CodeMirror verwendet kein festes Raster, darunter
leidet die Performanz. Da wir jedoch darauf angwiesen sind, müssen wie diese Einbußen in der
Geschwindigkeit akzeptieren. Seit Version 3.0 welche am 10.12.2012 erschien, ist es möglich
Textteile zu durch HTML- Widgets zu substituieren. Dadurch ist es möglich Isabelle Sonderzeichen
welche durch ASCII Sequenzen wie beispielsweise \texttt{\textbackslash\textless
rightarrow\textgreater} für das Zeichen $\rightarrow$ repräsentiert werden direkt im Editor zu
ersetzen, sodass der bearbeitete Text valider Isabelle-Code bleibt, die Darstellung hingegen der
eines LaTeX-Dokuments entspricht.

\end{description}

Weitere Editoren existieren zwar, scheiden aber alle aus, da die meisten nicht einmal die Hälfte der
oben formulierten Anforderungen erfüllen.

\subsection{Client-Modell}

\TODO{Client-Modell}