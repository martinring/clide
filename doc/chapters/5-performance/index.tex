\chapter{Bewertung und Ausblick}

Das Ergebnis der Arbeit ist ein innovatives, funktionstüchtiges Werkzeug, das in der Lage ist dem
Nutzer bei der Arbeit mit Beweisdokumenten eine sinnvolle und produktive Unterstützung zu gewähren.
Als Maßstab für die Einordnung der Ergebnise bietet sich ein Vergleich mit dem bisherigen Ansatz
Isabelle/jEdit an. 

Dabei ist allerdings zu beachten, dass es sich bei diesem Projekt auch um eine
Machbarkeitsstudie für die Umsetzung einer webbasierten IDE handelt.

“a mature programmer’s text editor with hundreds of person-years of development behind it”



Nicht vollständig (Machbarkeitsstudie)

Aber doch eigentlich echt geil

Schwierigkeiten wegen des Zusammenspiels der Vielen Bibliotheken

Was noch zu machen ist
 
 - Volle Funktionalitäten

 - Usability

 - Browserkompatibilität

Wo weiter gemacht werden kann

 - IDE Als Ansatz für andere Sprachen

 - Schnittstelle veralgemeinern

 - SML, Haskell, ...