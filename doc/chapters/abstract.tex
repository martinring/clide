\section*{\abstractname}

2010 wurde für den interaktiven Theorembeweiser Isabelle die Isabelle/Scala Schnittstelle eingeführt
sowie die Beispielanwendung Isabelle/jEdit als Entwicklungsumgebung für Isabelle entwickelt. Dabei
fiel die Wahl auf jEdit, weil der Aufwand für die Integration gering gehalten werden sollte. In
diesem Projekt konzentrieren wir uns nun auf die Ausgestaltung einer neuartigen
Entwicklungsumgebung. Dabei werden modernste Web-Techniken miteinander verknüpft und dadurch eine
ganz neue Art des Umgangs mit Beweisdokumenten geschaffen. Durch die Kombination des Webframeworks
Play, dem aktuellen Entwurf des HTML5-Standards und dem Isabelle/Scala-Layer als Schnittstelle zur
mächtigen Isabelle-Plattform entsteht ein Werkzeug, das verwendet werden kann, um ohne jeglichen
Konfigurationsaufwand für den Nutzer von einem beliebigen Rechner mit einem aktuellen Browser
Beweisdokumente bearbeiten zu können, ohne dass auf dem Rechner die gesamte Isabelle-Plattform
installiert sein muss. Es entstehen neue Konzepte wie serverseitiges Syntax-Highlighting und die
verzögerte Einbindung von auf dem Server ermittelten Beweiszuständen im Browser.