\chapter{Einführung}

\section{Motivation}

Die Benutzerschnittstellen interaktiver Theorembeweiser sind seit längerem ein wichtiger Teil der
Forschung im Gebiet der formalen Beweissysteme. Lange Zeit wurde hier die Interaktion durch ein
sequenzielles \textit{\acr{repl}}-Modell dominiert. Für das verbreitete Isabelle-System ist das
Standard-Werkzeug der \textit{Proof General}, welcher auf \textit{emacs} basiert und damit in einer
mächtigen Entwicklungsumgebung lebt und große Verbesserungen gegenüber den bisherigen Tools bietet.
Das grundsätzliche Interaktionsmodell des Proof General wird in \cite{iscala}, S. 2 jedoch immer
noch als sequenziell beschrieben (\textit{\glqq one command after another\grqq}).

\subsection{Isabelle/Scala}

2010 wurde von Makarius Wenzel in \cite{iscala} die \textit{Isabelle/Scala} Schnittstelle
eingeführt, die es erlaubt, in der Sprache Scala über ein einfaches Dokumentenmodell auf die
Isabelle-Plattform zuzugreifen. Scala besitzt die nötige Flexibilität, um die relevanten Teile von
Isabelle, das in Standard ML implementiert wurde, abzubilden. Da Scala darüber hinaus in der
\acr{jvm} lebt, bieten sich vielfältige Möglichkeiten, die Schnittstelle zu integrieren. Unter
anderem existieren für die \acr{jvm} viele ausgereifte Entwicklungsumgebungen.

Zusammen mit der Schnittstelle wurde zunächst die Isabelle/jEdit Umgebung eingeführt. Die Wahl fiel
dabei auf jEdit, da es über eine einfache Plugin-Achitektur verfügt und die Integration das Projekt
nicht zu sehr ablenken sollte. Dadurch wurden auch die Einschränkungen von jEdit gegenüber mächtigen
Entwicklungsumgebungen wie Eclipse, Netbeans oder IntelliJ akzeptiert. Isabelle/Scala wurde dabei
aber bewusst so konzipiert, dass es möglichst universell in eine beliebige Umgebung integriert
werden kann.

Eine Integration der Plattform in eine der großen Entwicklungsumgebungen ist zweifelsohne ein
wichtiges Projekt für ambitionierte Nutzer, birgt aber weiterhin den Nachteil, dass die komplexe,
große und ressourcenhungrige Isabelle-Plattform zunächst lokal auf dem Rechner installiert und vor
allem konfiguriert werden muss, was für Einsteiger eine Hürde darstellt und damit besonders für die
Lehre nicht optimal ist.

\subsection{HTML5}

Durch die Entwicklung der letzten Jahre im Bereich der Webprogrammierung, insbesondere der
Bemühungen des W3C um eine drastische Erweiterung der Möglichkeiten von HTML im zukünftigen
HTML5-Standard, ist es heute möglich Anwendungen, wie sie früher nur auf dem Desktop denkbar waren,
im Browser zu implementieren. Durch immer effizientere JavaScript-Engines und wachsende
Unterstützung für hardwarebeschleunigtes Rendering von Webseiten, wird die Webprogrammierung immer
populärer und so hat beispielsweise Microsoft in der aktuellen Version des hauseigenen
Betriebssystems sogar eine HTML/JS-basierte API zur Programmierung von Desktopanwendungen
integriert.

Ein Thema, das in Zeiten von Netbooks, Smartphones und Tablet PCs immer mehr an Bedeutung gewinnt,
ist die Portabilität und Mobilität von Daten und Anwendungen. Der Hype des \textit{Cloud Computing}
flaut nicht ab und der Erfolg der einschlägigen Webdienste (\textit{Dropbox}, \textit{Google Docs},
\textit{Instragram}, etc.) bestätigt den Bedarf. Durch neue Techniken wie WebSockets werden die
Möglichkeiten solche Dienste zu verwirklichen immer geeigneter.

\section{Aufgabenstellung}

In diesem Projekt soll durch Kombination der Isabelle/Scala Schnittstelle und den neusten
Webtechniken eine mobile, plattformunabhängige Entwicklungsumgebung, speziell für die Arbeit mit
Isabelle entstehen. Die Umgebung ist dabei der Browser. Die IDE soll ohne Konfigurationsaufwand für
den Endnutzer auskommen und dabei trotzdem signifikante Verbesserungen gegenüber Isabelle/jEdit
bieten, die im Folgenden aufgeführt werden:

\begin{itemize}
  \item Geringer Ressourcenbedarf auf Clientseite,
  \item verbesserte Visualisierung schon während der Bearbeitung,
  \item verbesserte Integration der Ausgaben in die Visualisierung,
  \item volle Mobilität (Unabhängigkeit vom Client) sowie
  \item Mehrbenutzerfähigkeit.
\end{itemize}

Das Projekt versteht sich dabei als Machbarkeitsstudie, da der volle Funktionsumfang einer
ausgereiften Entwicklungsumgebung hunderte Personenjahre Entwicklungszeit benötigen würde.

\section{Anmerkungen}

Die beiliegende Software kann unter Windows, OS X und Linux installiert werden. Genauere
Anweisungen sind in Anhang\,\ref{sec:install} zu finden.

Die Quellen sind sowohl auf dem beiliegenden USB-Stick, als auch im Internet zu finden:

\begin{quote}
\textit{http://www.github.com/martinring/clide}
\end{quote}